\setcounter{equation}{0}
\section{Безразмерни течения}
Ще разгледаме вискозни течения с непроменлива динамична вискозност $\mu$.
Същото предполагаме и за масовите сили $\mathbf{g}$.
Експериментални изследвания върху течения с модели/макети могат да служат за качествено/количествено характеризиране на по-големи обекти, които на практика могат да се ползват (напр. кораби, самолети).
За тази цел се използва обезразмеряване.

\subsection{Безразмерен запис на уравнения на течения}
Нека разгледаме системата от уравнения на Навие-Стокс и уравнението на непрекъснатостта
\begin{gather}
	\pdv{\mathbf{v}}{t} + \mathbf{v} \cdot \grad \mathbf{v} = \mathbf{g} - \frac{1}{\rho}\grad p + \nu \laplacian \mathbf{v} \\\nonumber
	\div \mathbf{v} = 0
\end{gather}
Разглеждаме тяло с характерна дължина $l$. Правим смяна на координатите, като искаме да разпишем уравненията в следната система
\begin{equation}
	\xi = \frac{x}{l},\, \eta = \frac{y}{l},\, \zeta = \frac{z}{l},\, \tau = \frac{t}{\frac{l}{\nu}}
\end{equation}
Въвеждаме безразмерни функции
\begin{equation}
	\mathbf{u} = \frac{l}{\nu} \mathbf{v},\, \Pi = \frac{l^2}{\nu^2} \frac{p}{\rho},\, \bm{\gamma} = \frac{l^3}{\nu^2} \mathbf{g}
\end{equation}
Трябва да ги запишем като функции на новите координати. За да сведем уравненията използваме, че
\begin{align}
	&\pdv{\mathbf{v}}{t} = \pdv{(\frac{\nu}{l}\mathbf{u})}{\tau}\dv{\tau}{t} = \frac{\nu^2}{l^3} \pdv{\mathbf{u}}{\tau} \\
	&\mathbf{v} \cdot \grad \mathbf{v} = 
		\frac{\nu}{l} \mathbf{u} \cdot \grad \frac{\nu}{l} \mathbf{u} = 
		\frac{\nu^2}{l^2} \mathbf{u} \cdot (\pdv{\mathbf{u}}{\xi}\dv{\xi}{x} + \pdv{\mathbf{u}}{\eta}\dv{\eta}{y} + \pdv{\mathbf{u}}{\zeta}\dv{\zeta}{z}) =
		\frac{\nu^2}{l^3} \mathbf{u} \cdot (\pdv{\mathbf{u}}{\xi} + \pdv{\mathbf{u}}{\eta} + \pdv{\mathbf{u}}{\zeta}) \\
	& \div \mathbf{v} = 0 \iff \pdv{u_x}{\xi} + \pdv{u_y}{\eta} + \pdv{u_z}{\zeta} = 0 \\
	& \nu \laplacian \mathbf{v} = 
		\nu \div \frac{\nu}{l} (\pdv{\mathbf{u}}{\xi}\dv{\xi}{x} + \pdv{\mathbf{u}}{\eta}\dv{\eta}{y} + \pdv{\mathbf{u}}{\zeta}\dv{\zeta}{z}) = 
		\frac{\nu^2}{l^2} (\pdv[2]{\mathbf{u}}{\xi}\dv{\xi}{x} + \pdv[2]{\mathbf{u}}{\eta}\dv{\eta}{y} + \pdv[2]{\mathbf{u}}{\zeta}\dv{\zeta}{z}) = \\\nonumber
		&\frac{\nu^2}{l^3} (\pdv[2]{\mathbf{u}}{\xi} + \pdv[2]{\mathbf{u}}{\eta} + \pdv[2]{\mathbf{u}}{\zeta}) \\
	&\frac{1}{\rho}\grad p = 
		\grad \frac{p}{\rho} =
	 	\frac{\nu^2}{l^2} (\pdv{\Pi}{\xi}\dv{\xi}{x} + \pdv{\Pi}{\eta}\dv{\eta}{y} + \pdv{\Pi}{\zeta}\dv{\zeta}{z}) =
		\frac{\nu^2}{l^3} (\pdv{\Pi}{\xi} + \pdv{\Pi}{\eta} + \pdv{\Pi}{\zeta})
\end{align}
Съкращаваме и получаваме системата
\begin{gather}
	\dv{\mathbf{u}}{\tau} = \bm{\gamma} - \grad \Pi + \laplacian \mathbf{u} \\\nonumber
	\div \mathbf{u} = 0
\end{gather}
Тук операторите са спрямо новите ни променливи $\xi, \eta, \zeta$, като вече единицата за дължина е характерната дължина на тялото, т.е. $l$.

\subsection{Подобни координати}
Нека имаме две подобни тела със съответни характерни дължини $l_1$, $l_2$ - те ще определят линейния мащаб за двете задачи. 
Изразяваме съответно течения с кинематични вискозности $\nu_i = \frac{\mu_i}{\rho_i}$ в координати $x_i,y_i,z_i,t_i, \quad i=1,2$.
Да забележим, че $[\nu_i]=\frac{L^2}{T}$, а $[l_i]=L$.
Така за мащаб по времето може да вземем $\frac{l_i^2}{\nu_i},\, i=1,2$. 
За обезразмерени уравнения въвеждаме координати
\begin{equation}
	\xi_i = \frac{x_i}{l_i},\, \eta_i = \frac{y_i}{l_i},\, \zeta_i = \frac{z_i}{l_i},\, \tau_i = \frac{t_i}{\frac{l_i^2}{\nu_i}}, \quad i=1,2
\end{equation}
Подобни координати на двете течения наричаме тези, за които всички двойки безразмерни величини съвпадат.
След тези преобразузавания и двете безразмерни тела имат характерни дължини $1$ и са геометрически еднакви. 
\subsection{Подобие при вискозни течения}
Иска ни се с едно течение да оприличим друго - както например имаме геометрично подобие на фигури и сме извели някакво свойство/количество за една от тях, лесно може да го получим за другата.
Ще казваме, че две течения са подобни, ако са около подобни тела и стойностите на техните хидромеханични величини в подобни координати са еднакви с точност до константен множител (не задължително еднакъв за различните величини). 
И тъй нека имаме две течения със съответни величини 
\begin{equation}
	l_i, \mathbf{v}_i, \mathbf{g}_i, \nu_i, \frac{p_i}{\rho_i}, \quad i=1,2
\end{equation}
Теченията имат и безразмерни уравнения и нека разгледаме задачата за обтичане по тяло. 
В безразмерни координати телата се изобразяват в "единично" тяло със същата форма.
Нека бележим границата му с $S$. 
\begin{enumerate}
	\item Трябва да са изпълнени граничните условия по границата на тялото - $\mathbf{u}_1\vert_S = \mathbf{u}_2\vert_S = \mathbf{0}$.
	\item Трябва да са изпълнени граничните условия в безкрайност - $\mathbf{u}_1\vert_\infty = \mathbf{U}_1,\, \mathbf{u}_2\vert_\infty = \mathbf{U}_2$.
\end{enumerate}
Достатъчно е да са изпълнени следните равенства за безразмерните величини - $\mathbf{u}_1 = \mathbf{u}_2,\, \Pi_1 = \Pi_2$.
За да са изпълнени е достатъчно двете безразмерни уравнения да съвпадат, както и граничните условия да съвпадат.

Очевидно граничните условия по границата на тялото са едни и същи.
За да съвпадат тези в безкрайност, то трябва
\begin{equation}
	\frac{\mathbf{V}_1 l_1}{\nu_1} = \mathbf{U}_1 = \mathbf{U}_2 = \frac{\mathbf{V}_2 l_2}{\nu_2}
\end{equation}

За да съвпадат уравненията ще трябва
\begin{equation}
	\frac{\mathbf{g}_1 l_1^3}{\nu_1^2} = \bm{\gamma}_1 = \bm{\gamma}_2 = \frac{\mathbf{g}_2 l_2^3}{\nu_2^2}
\end{equation}

Последните две уравнения са векторни. Изпълнени са точно когато съответните вектори от двете страни са колинеарни и
\begin{gather}
	\label{ReSame} \frac{\norm{\mathbf{V}_1} l_1}{\nu_1} = \frac{\norm{\mathbf{V}_2} l_2}{\nu_2}\\
	\frac{\norm{\mathbf{g}_1} l_1^3}{\nu_1^2} = \frac{\norm{\mathbf{g}_2} l_2^3}{\nu_2^2}
\end{gather}
Обикновено се взима еквивалента система уравнения - \eqref{ReSame} и 
\begin{equation}
	\label{FrSame} \frac{\norm{\mathbf{V}_1}^2}{\norm{\mathbf{g}_1} l_1} = \frac{\norm{\mathbf{V}_2}^2}{\norm{\mathbf{g}_2} l_2}
\end{equation}

Веднага може да видим, че ако гравитационните сили са същите за $2$ подобни (но не еднакви) тела, то няма как в една и съща среда да имат подобни течения.
Да допуснем противното и изразим $\norm{\mathbf{V}_2} = \frac{\norm{\mathbf{V}_1} l_1}{l_2}$ от \eqref{ReSame}.
Но сега след съкращаване в \eqref{FrSame} получаваме $\frac{l_1^2}{l_2^2} = 1$, откъдето $l_1 = l_2$.

\subsection{Безразмерни характерни числа}
На някои места се срещат само като безразмерни числа, но числата в математиката са си безразмерни.
На други места се срещат като безразмерни величини, безразмерни комплекси. 
Това са числа, които често определят качествената характеристика на потока.
Рядко могат да съвпаднат всичките за модела и реалния обект.
Затова инжинерите преценяват кои явления е най-важно да бъдат моделирани, тези които имат най-голям ефект върху движението на крупния обект.

\subsubsection{Основни безразмерни характерни числа}
Числото на Рейнолдс $Re = \frac{\norm{\mathbf{V}} l}{\nu}$ има значение само за вискозни флуиди, т.к. за идеалните $\nu = 0$, т.е. $Re = \infty$.
То представлява отношението на инерчните сили към вискозните сили.
Има голямо значение при изследването на флуидното съпротивление.

Числото на Фруд $Fr = \frac{\norm{\mathbf{V}}}{\sqrt{\norm{g}} l}$ има смисъл и за невискозни флуиди.
Квадратът му представлява отношението на инерчните сили към гравитационните сили. 
Има голямо значение при изследване на влиянието на вълни над тела - тогава гравитационната сила оказва съществено влияние.
Обратно - ако цялото тяло е потопено във флуид, то действието на гравитацията се състои в добавянето на налягане и тогава $Fr$ е без голямо значение.
Изключение би било, ако тялото се намира на граница на две фази.
То е най-широко разпространеното безразмерно число в механиката на флуидите.

Така за да са подобни две течения, трябва числата им на Рейнолдс и Фруд да съвпадат. 
Оказва се, че при моделирането с реални макети това не е елементарно. 
Може да допуснем, че гравитационната сила е еднаква, т.к. разликата при нея ще е минимална - а даже и в зависимост къде правим експеримента може да е никаква.
Ако искаме да използваме макет с малък мащаб, то намаляваме $l$ да кажем $25$ пъти, за да остане $Fr$ непроменено, то ще трябва да увеличим относителната скорост $5$ пъти.
Още повече, заради това ще променим числото на Рейнолдс. За да предотвратим това, е необходимо да използваме течност с кинематичен вискозитет $125$ пъти по-малък от тази в която ще се използва истинсият обект.
Но например в корабостроенето модел само $1:25$ не би бил малък. Ако се моделира голям кораб, макета би бил $10-12m$.
Има някои течности, които при подходящи температури имат $5-10$ пъти по-нисък кинематичен вискозитет от водата - например живак, охладителна течност, толуен.
Течности с необходимия вискозитет практически не съществуват.
Понякога е възможно хидродинамични тестове да се правят във въздушна среда, т.к. кинематичният вискозитет на въздуха е близо $15$ пъти по-малък от този на водата.
Обратното също е възможно - малки макети при малки скорости на водата могат да имат подобни течения на големи тела с голяма относителна скорост спрямо въздуха.
Едно от големите предимства на развилите се последните няколко десетилетия компютърни системи за флуидни симулации е, че се извършват от компютъра и дори математическият модел да не улови изцяло физически случващото се, то симулацията може да извърши изчисления за течение, за което не можем физически да намерим подобно, с което лесно да се борави.

Числото на Струхал $St = \frac{\norm{\bm\Omega} l}{\norm{\mathbf{V}}}$ има смисъл и за невискозни флуиди.
То представлява отношението на вихровото ускорение към адвективното ускорение.
За безвихрови течения е $0$. Характеризира турболенти потоци и вихрови осцилации, получаващи се зад обекти, наричано вихрова следа.
Във връзка с това има и друга дефиниция, свързваща го директно с честотата на такива осцилации.
Рошко провел серия експерименти в началото на $50$-те години във връзка с обтичане по цилиндър.
При $500 < Re < 500 000$, числото на Струхал практически остава непроменено.
Повечето риби и летящи животни се движат с $0.2 < St < 0.4$.

Числото на Мах $M = \frac{\norm{\mathbf{V}}}{c}$, където $c$ е скоростта на звука в средата.
Квадратът му представлява отношението на итерчните сили към силите на свиваемост.
С него може да се мери изентропното отклонение от закона за несвиваемост.
Течения при $M < 1$ се наричат субзвукови, а при $M > 1$ - суперзвукови.

\subsubsection{Безразмерни характерни числа при топлопроводни потоци}
Числото на Екерт $Ec = \frac{\norm{\mathbf{V}}^2}{c_p(T_w - T_f)}$, $T_f,\, T_w$ са съответно температурите на свободния поток и стените.
То представлява отношението на кинетичната към топлинната енергия.

Числото на Прантл $Pr = \frac{c_p \mu_0}{k_0}$.
Това е отношението на разсейването на импулса към топлинното разсейване.

\subsubsection{Безразмерни характерни числа при повърхностни напрежения}
Числото на Вебер $We = \frac{\norm{\mathbf{V}}^2 \rho l}{\sigma}$ измерва отношението на инерчните сили към тези на повърностното напрежение.
При големи числа на Вебер капки лесно се деформират при сбъсъци с твърди повърхности.

Числото на Бонд $Bo = \frac{\norm{\mathbf{g}} \rho l^2 }{\sigma}$ е отношението на гравитационните сили към тези на повърностното напрежение.
При големи числа на Бонд гравитационните сили оказват много по-голямо влияние при течения около повърхността на флуида, отколкото повърхностното напрежение.

Капилиарното число $Ca = \frac{\norm{\mathbf{V}} \mu}{\sigma}$ представлява отношението на вискозното напрежение към повърностното напрежение.
Бавни течения през порести среди или тясни тръби се характеризират с ниски капилярни числа и образуване на мехурчета.
