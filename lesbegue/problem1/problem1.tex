\documentclass[bulgarian, 12pt]{article}
\usepackage[
  a4paper,
  includeheadfoot,
  margin = 1.5 cm]
{geometry}

% Fonts
\usepackage[T2A]{fontenc}
% \usepackage{tempora}
\usepackage[utf8]{inputenc}
\usepackage{bm}

% % Languages
% \usepackage[base]{babel}
% % Some languages define these commands, so you have to write this to protect yourself from namespace collision errors
% \AfterBabelLanguage{bulgarian}{%
%   \let\sh\relax\let\ch\relax\let\tg\relax
%   \let\arctg\relax\let\arcctg\relax
%   \expandafter\let\expandafter\th\csname ltx@th\endcsname
%   \let\ctg\relax\let\cth\relax\let\cosec\relax
% }
\usepackage[bulgarian, english]{babel}
\usepackage{hyphenat}

% Indent first line in paragraph
\usepackage{indentfirst}

% Place tags on the left
\usepackage[leqno]{amsmath}

% Better math
\usepackage{amsbsy}
\usepackage{amssymb}
\usepackage{mathtools}
\usepackage{comment}
\usepackage{mathptmx}
\usepackage[makeroom]{cancel}

% Better theorems
\usepackage{amsthm}

% SI units
%\usepackage{siunitx}

% Derivative notations
\usepackage{physics}
%\usepackage{derivative}

\usepackage{blindtext}
\usepackage{scrextend}

\title{\textbf{Задача 1} относно обобщението на мярка на Лебег за $\mathbb{R}$}
\author{Калоян Стоилов}

\begin{document}
\selectlanguage{bulgarian}
\maketitle
\begin{labeling}{задача}
\item [(а)] Нека $\{I_k\}$ и $\{J_k\}$
  са две разбивания. Трябва да проверим, че:
  \[
    m_I(S) = \sum_{k=1}^{\infty} m(S \cap I_k) = \sum_{l=1}^{\infty} m(S \cap J_l) = m_J(S)
  \]
  % От дефинициите на съответените мерки, може да пресметнем:
  % \begin{eqnarray*}
  %   m_I(J_l) = \sum_{k=1}^{\infty} m(J_l \cap I_k)  \\
  %   m_J(I_k) = \sum_{l=1}^{\infty} m(I_k \cap J_l)
  % \end{eqnarray*}
  Понеже $S \subseteq \mathbb{R}$, то $S = S \cap \mathbb{R}$ и от $\mathbb{R} = \bigcup\limits_{k=1}^{\infty} I_k = \bigcup\limits_{l=1}^{\infty} J_l$ следва:
  \begin{eqnarray*}
    S = S \cap \bigcup\limits_{k=1}^{\infty} I_k = \bigcup\limits_{k=1}^{\infty} S \cap I_k \\
    S = S \cap \bigcup\limits_{l=1}^{\infty} J_l = \bigcup\limits_{l=1}^{\infty} S \cap J_l
  \end{eqnarray*}
  Използвайки горните представяния на S, както и пълната адитивност на m за интервали и че интервалите от разбиванията имат само общи краища:
  \begin{eqnarray*}
    m_I(S) = \sum_{k=1}^{\infty} m(S \cap I_k) = \sum_{k=1}^{\infty} m((\bigcup\limits_{l=1}^{\infty} S \cap J_l) \cap I_k) = \sum_{k=1}^{\infty} m(\bigcup\limits_{l=1}^{\infty} S \cap J_l \cap I_k) = \sum_{k=1}^{\infty} \sum_{l=1}^{\infty} m(S \cap J_l \cap I_k) = \\
    \sum_{l=1}^{\infty} \sum_{k=1}^{\infty} m(S \cap J_l \cap I_k) = \sum_{l=1}^{\infty} m(\bigcup\limits_{k=1}^{\infty} S \cap I_k \cap J_l) = \sum_{l=1}^{\infty} m((\bigcup\limits_{k=1}^{\infty} S \cap I_k) \cap J_l) = \sum_{l=1}^{\infty} m(S \cap J_l) = m_J(S)
  \end{eqnarray*}
\item [(б)] Трябва да проверим трите точки от дефиницията на $\sigma$-алгебра:
  \begin{itemize}
      % \item $m(\emptyset) = \sum_{k=1}^{\infty} m(\emptyset \cap I_k) = \sum_{k=1}^{\infty} m(\emptyset) = \sum_{k=1}^{\infty} 0 = 0$
    \item $\emptyset$ е измеримо, понеже всичките $\{\emptyset \cap I_k\}$ са измерими
    \item $S_1,S_2\ldots \subseteq \mathbb{R}, \quad S_1,S_2\ldots \text{- измерими} \implies {S_l \cap I_k} \text{- измерими} \implies {\bigcup\limits_{l=1}^{\infty}(S_l \cap I_k)} \text{- измерими} \implies {(\bigcup\limits_{l=1}^{\infty}S_l) \cap I_k} \text{- измерими} \implies \bigcup_{l=1}^{\infty}S_l \text{- измеримо}$
    \item $S \subseteq \mathbb{R}, \quad S \text{ - измеримо} \implies \mathbb{R} \setminus S \subseteq \mathbb{R}, \{S \cap I_k\} \text{ - измерими} \implies \{I_k \setminus (S \cap I_k)\} \text{ - измерими} \implies \{I_k \cap (\mathbb{R} \setminus S)\} \text{ - измерими} \implies \mathbb{R} \setminus S \text{ - измеримo}$
  \end{itemize}
\item [(в)] Ако имаме непресичащи се $\{S_l\}$ то:
  \[
    m(\bigcup_{l=1}^{\infty} S_l) = \sum_{k=1}^{\infty} m((\bigcup_{l=1}^{\infty} S_l) \cap I_k) = \sum_{k=1}^{\infty} m(\bigcup_{l=1}^{\infty} (S_l \cap I_k)) \overset{\text{адит. m}}{=} \sum_{k=1}^{\infty} \sum_{l=1}^{\infty} m(S_l \cap I_k) = \sum_{l=1}^{\infty} \sum_{k=1}^{\infty} m(S_l \cap I_k) = \sum_{l=1}^{\infty} m(S_l)
  \]
\end{labeling}
Безкрайните суми разменяхме от неотрицателността на мярката $m$ и в този случай сходимост е еквивалента на абсолютна сходимост.
\end{document}
