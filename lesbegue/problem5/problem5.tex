\documentclass[bulgarian, 12pt]{article}
\usepackage[
  a4paper,
  includeheadfoot,
  margin = 1.5 cm]
{geometry}

% Fonts
\usepackage[T2A]{fontenc}
% \usepackage{tempora}
\usepackage[utf8]{inputenc}
\usepackage{bm}

% % Languages
% \usepackage[base]{babel}
% % Some languages define these commands, so you have to write this to protect yourself from namespace collision errors
% \AfterBabelLanguage{bulgarian}{%
%   \let\sh\relax\let\ch\relax\let\tg\relax
%   \let\arctg\relax\let\arcctg\relax
%   \expandafter\let\expandafter\th\csname ltx@th\endcsname
%   \let\ctg\relax\let\cth\relax\let\cosec\relax
% }
\usepackage[bulgarian, english]{babel}
\usepackage{hyphenat}

% Indent first line in paragraph
\usepackage{indentfirst}

% Place tags on the left
\usepackage[leqno]{amsmath}

% Better math
\usepackage{amsbsy}
\usepackage{amssymb}
\usepackage{mathtools}
\usepackage{comment}
\usepackage{mathptmx}
\usepackage[makeroom]{cancel}

% Better theorems
\usepackage{amsthm}

% SI units
%\usepackage{siunitx}

% Derivative notations
\usepackage{physics}
%\usepackage{derivative}

\usepackage{blindtext}
\usepackage{scrextend}

\title{\textbf{Задача 5} относно конволюцията в $L(\mathbb{R, m})$}
\author{Калоян Стоилов}

\begin{document}
\selectlanguage{bulgarian}
\maketitle
Надолу $f, g \in L(\mathbb{R, m})$ . $(\mathbb{R}, m)$ е $\sigma$-крайно, понеже може да го представим като изброимо обединение от интервали с дължина 1. Тогава може да ползваме теоремите на Тонели и Фубини за $(\mathbb{R} \cross \mathbb{R}, m \cross m)$.\\
\begin{labeling}{задача}
  %\item[(а)]  $\cap(f)_x(t) = f(x-t), е измерима, понеже е в сила:
  % \begin{eqnarray*}
  %   \{t \in \mathbb{R} | \cap(f)_x(t) > c \} = \{t \in \mathbb{R} | f(x-t) > c \} =
  % \end{eqnarray*}

\item [(a), (б), (в)] Доказвайки (в), получаваме като следствие (б), понеже $h \in L(\mathbb(R), m) \iff |h| \in L(\mathbb{R}, m) \iff \norm{h}_L < \infty$. Забелязваме, че:
  \begin{align*}
    & \norm{h}_L = \int_{\mathbb{R}} |h(x)| \dd m(t) = \int_{\mathbb{R}} \int_{\mathbb{R}} |f(x-t)g(t)| \dd m(t) \dd m(x) \leq \int_{\mathbb{R}} \int_{\mathbb{R}} |f(x-t)||g(t)| \dd m(t) \dd m(x) \overset{\text{Тонели}}{=} \\
    & \int_{\mathbb{R}} \int_{\mathbb{R}} |f(x-t)| \dd m(x) |g(t)| \dd m(t) \overset{\text{упътване}}{=}
    \int_{\mathbb{R}} \int_{\mathbb{R}} |f(x)| \dd m(x) |g(t)| \dd m(t) = \norm{f}_L \int_{\mathbb{R}} |g(t)| \dd m(t) = \\
    & \norm{f}_L \norm{g}_L < \infty \cdot \infty = \infty
  \end{align*}
  Наистина са изпълнени усливията за теоремата на Фубини, понеже може да видим, че:
  \begin{align*}
    &\int_{\mathbb{R} \cross \mathbb{R}} |f(x-t)g(t)| \dd m(x) \cross m(t) \overset{\text{Тонели}}{=} \int_{\mathbb{R}} \int_{\mathbb{R}} |f(x-t)| |g(t)| \dd m(x) m(t) = \\
    & \int_{\mathbb{R}} \int_{\mathbb{R}} |f(x-t)| \dd m(x) |g(t)| \dd m(t) \overset{\text{вж. горе}}{<} \infty
  \end{align*}
  Така имаме, че $|f(x-t)g(t)|$ е интегруема върху $\mathbb{R} \cross \mathbb{R}$, откъдето проекцията ѝ $h(x)$ е интегруема върху $\mathbb{R}$ за почти всяко $x$, т.е. имаме и (a).
  % За да може да приложим Тонели, ни трябва измеримостта на |f(x-t)g(t)|, но тя следва от тази на f(x-t)g(t).
\end{labeling}
\end{document}
