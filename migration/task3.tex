\begin{frame}[t]{Задача 3}
Въвеждаме следните означения:\\
$\eta_{1} = \alpha_{1} a_{1} (1-d_{1}) + \alpha_{2} a_{2} (1-d_{2}),$ \\ 
$\eta_{2} = 1 + \alpha_{1} \alpha_{2} a_{1} a_{2} (1 - d_{1} - d_{2})$ \\
Да се покаже, че при $\eta_{1} < \eta_{2} < 2$ популациите и в двете области изчезват, т.е. $\lim\limits_{n \to \infty}x_{i}(n)=0, \enspace i=1,2$, независимо от началните условия.
\end{frame}