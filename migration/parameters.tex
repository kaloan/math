%\section{Параметрите и стойностите им}

\begin{frame}[t]{Параметрите и стойностите им}

  Параметрите трябва да са такива, че моделът да е дефиниран за всяко $n \geq 0$, както и при задаване на правдоподобни стойности - неотрицателни числа като начални условия, то да имаме и $x_{i}(n) \geq 0, n \geq 0, \; i=1,2$. Също така трябва да отчетем, че при някои стойности на параметри моделът може да се изроди в друг.

\end{frame}

\begin{frame}[t]{Параметрите и стойностите им}

  Желаем това от параметрите, тъй като при $b_{i}=0$ моделът се изражда в експоненциален, а при $a_{i}=0$ просто $x_{i}(n+1)=0$. Желаем неотрицателността им, тъй като при $b_{i} < 0$, имаме $x_{i} = 1/b_{i}$ води до недефинираност на модела, а при $a_{i} < 0 $, то $sgn(x_{i}(n+1)) = - sgn(x_{i}(n))$, поне при $x_{i} < 1/b_{i}$. Също така при $b_{i} < 0$, ако $x_{i} > 1/b_{i}$, при $a_{i} > 0 $, то отново $sgn(x_{i}(n+1)) = - sgn(x_{i}(n))$.

\end{frame}

\begin{frame}[t]{Параметрите и стойностите им}

  Подобни разсъждения може да направим и при динамиката за n нечетно. Нека $i \in \{1, 2\}$, a $j \in \{1, 2\}, j \neq i$. Фиксирайки $x_{i}(0)=0$, то $x_{i}(1)=x_{i}(0)\frac{a_{i}}{1+b_{i} x_{i}(0)}=0$, откъдето имаме:
  \[x_{i}(2)=d_{j} x_{j}(1)\frac{\alpha_{j}}{1+\beta_{j} x_{j}(1)}\]
  \[x_{j}(2)=(1-d_{j}) x_{j}(1)\frac{\alpha_{j}}{1+\beta_{j} x_{j}(1)}\] Тогава $\beta_{j} > 0$, трябва $d_{j}\alpha_{j}>0$ и $(1-d_{j})\alpha_{j}>0$, откъдето $\alpha_{j}>0$, а $d_{j} \in [0,1]$, като за нашите цели $d_{j} \neq 0, 1$.

\end{frame}
