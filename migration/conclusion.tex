\begin{frame}[t]{Заключение}

Доказахме следните твърдения:
\begin{enumerate}
\item При адекватна параметризация, моделът е правдоподобен.
\item Ако има индивиди в едното местообитание, то има и в другото.
\item Популациите винаги са ограничени.
\item Възможно е видът да изчезне и от двете местообитания.
\item Изглежда, когато популациите не изчезват, гъстотите клонят към устойчив цикъл.
\item Изглежда, че приближаването към устойчивите точки за всеки сезон е монотонно.

\end{enumerate}

\end{frame}