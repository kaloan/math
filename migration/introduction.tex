%\section{Динамика на Бевъртън-Холт}

\begin{frame}[t]{Динамика на Бевъртън-Холт}

  Динамиката на Бевертън-Холт е вид популационна динамика, където $x(n+1)=\frac{\lambda x(n)}{1+bx(n)}$. Моделът който ще разглеждаме, ще се базира на тази динамика. Предполагаме наличието на две местности, които обитава някой вид. Също така времето е разделено на периоди така, че се редуват:
  \begin{itemize}
    \item период на размножаване във всяка от териториите
    \item период на размножаване с миграция на индивиди
  \end{itemize}

\end{frame}

\begin{frame}[t]{Формулировка на миграционния модел}

  Когато n е четно, то имаме следната динамика:
  \[x_{1}(n+1)=x_{1}(n)\frac{a_{1}}{1+b_{1} x_{1}(n)}\]
  \[x_{2}(n+1)=x_{2}(n)\frac{a_{2}}{1+b_{2} x_{2}(n)}\]

  Когато n е нечетно, то тя е:
  \[x_{1}(n+1)=(1-d_{1})x_{1}(n)\frac{\alpha_{1}}{1+\beta_{1} x_{1}(n)} + d_{2} x_{2}(n)\frac{\alpha_{2}}{1 + \beta_{2} x_{2}(n)}\]
  \[x_{2}(n+1)=d_{1}x_{1}(n)\frac{\alpha_{1}}{1+\beta_{1} x_{1}(n)} + (1-d_{2}) x_{2}(n)\frac{\alpha_{2}}{1 + \beta_{2} x_{2}(n)}\]

  Стойностите на параметрите удовлетворяват:
  $\alpha_{i},  \beta_{i}, a_{i}, b_{i} > 0,$ $i=1,2$ и приемаме, че $a_{i} \neq \alpha_{i}, b_{i} \neq \beta_{i}, i=1,2$.
  
\end{frame}
