\begin{frame}[t]{Физиология на невроните}
  Поддържане на различна концентрация на йони в цитозола спрямо околната тъкан.
  \ce{K+} и някои белтъци с общ отрицателен заряд са с по-голяма концентрация в клетката,
  докато \ce{Na+} и \ce{Cl-} извън нея. Това води до създаването на тъй наречения (транс-)мембранен потенциал.
  Клетъчната мембрана се състои основно от липиди, чиято хидрофобна част се държи като диелектрик.
  Липидната част на мембраната всъщност се оказва кондензатор. 
  Предвижването на йони през мембраната става чрез помпи (активно) и канали (пасивно). 
  Основни са тъй наречените натриево-калиевите помпи, това са молекули \ce{Na-K-ATPase}. Те предвижват 3 \ce{Na+} йона навън и 2 \ce{K+} навътре. 
  Други важни помпи са натриево-калциевите, като предвижват 1 \ce{Ca2+} йон навън и 3 \ce{K+} навътре.
\end{frame}

\begin{frame}[t]{Причината за трансмембранния потенциал}
  \begin{figure}[htbp!]
    \centering
    \includegraphics[width=\textwidth,height=0.7\textheight,keepaspectratio]{membrane-potential.png}
    \caption{Йони в цитозола и междуклетъчното пространство. Фиг 1.7 от Wikipedia}
  \end{figure}
\end{frame}

\begin{frame}[t]{Физиология на невроните}
  \begin{figure}[htbp!]
    \centering
    \includegraphics[width=\textwidth,height=0.7\textheight,keepaspectratio]{electrical-recording.PNG}
    \caption{Измерване на потенциала. Фиг 1.7 от From Neuron to Brain}
  \end{figure}
\end{frame}

\begin{frame}[t]{Физиология на невроните}
  \begin{figure}[htbp!]
    \centering
    \includegraphics[width=\textwidth,height=0.7\textheight,keepaspectratio]{mri-recording.PNG}
    \caption{Измерване на активността на невроните в главния мозък чрез ЯМР. Фиг 1.9 от From Neuron to Brain}
  \end{figure}
\end{frame}

\begin{frame}[t]{Междуневронна комуникация}
  Предаване на сигнали между неврони става през синапсите. 
  При електричните синапси става директно, едната клетка "продължава" в другата.
  При химическите синапси се образуват везикули (балончета с вещества) в синаптичния цитозол, 
  които се придвижват до мембраната и освобождават вътрешността си в междуклетачното пространство. 
  В резултат следващия неврон може или да бъде възбуден, или инхибиран.
\end{frame}

\begin{frame}[t]{Междуневронна комуникация}
  \begin{figure}[htbp!]
    \centering
    \includegraphics[width=\textwidth,height=0.7\textheight,keepaspectratio]{synapse-types.PNG}
    \caption{Двата вида синапси. Фиг 5.1 от Neuroscience}
  \end{figure}
\end{frame}

\begin{frame}[t]{Междуневронна комуникация}
  \begin{figure}[htbp!]
    \centering
    \includegraphics[width=\textwidth,height=0.7\textheight,keepaspectratio]{neurotransmitters.PNG}
    \caption{Молекулите на някои невротрансмитери. Фиг 6.1 от Neuroscience}
  \end{figure}
\end{frame}

\begin{frame}[t]{Междуневронна комуникация}
  \begin{figure}[htbp!]
    \centering
    \includegraphics[width=\textwidth,height=0.7\textheight,keepaspectratio]{gangleon-integration.PNG}
    \caption{Интегриране от ганглийна клетка. Фиг 1.16 от From Neuron to Brain}
  \end{figure}
\end{frame}

\begin{frame}[t]{Нервен импулс}
  \begin{figure}[htbp!]
    \centering
    \includegraphics[width=\textwidth,height=0.7\textheight,keepaspectratio]{ac-current.PNG}
    \caption{Резултат от подаване на прав ток чрез електрод в цитозола на ретинална ганглийна клетка. Фиг 1.12 от From Neuron to Brain}
  \end{figure}
\end{frame}

\begin{frame}[t]{Причина за нервния импулс}
  Нервният импулс е резултат от каскадно действие на мембранните помпи. 
  В клетката излизат \ce{Ca2+} и влизат \ce{Na+}, което води до промяна на потенциала и активиране на \ce{Na-K-ATPase}.
  Това води до още промяна на потенциала в околност на натриево-калиевата помпа, което активира други около нея.
\end{frame}