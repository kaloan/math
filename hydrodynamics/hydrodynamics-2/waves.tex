\setcounter{equation}{0}
\section{Гравитационни вълни}
Хидродинамиката разглежда 3 вида вълни - повърхнинни, вътрешни и свивателни.
И в трите случая вълните са предвижващи се флуидни трептения.
При първия тип връщащи сили се оказват повърхностното напрежение и гравитацията,
при вторите - само гравитацията, а при третите - от свиваемостта на флуида.
Вълните на границата на въздух-вода са добър пример за повърхнинни вълни и се наричат водни вълни.

\subsection{Основни понятия}
Нека разглеждаме координатна система с абсциса $x$ и ордината абсциса $z$.
Синусуидална бягаща вълна се представя с уравнението
\begin{equation}
  z(x,t) = A \cos \left[\frac{2 \pi}{\lambda}(x - c t)\right]
\end{equation}
Тук $A$ е амплитудата на вълната - най-голямото отдалечаване от положението.
Ако фиксираме времето $t$, менейки $x$, то имаме следваща амплитуда при:
\begin{equation*}
  \frac{2 \pi}{\lambda}((x + \lambda) - c t) = \frac{2 \pi}{\lambda}(x - c t) + 2 \pi
\end{equation*}
$\lambda$ е дължината на вълната - разстоянието между две съседни амплитуди.
$k = \frac{2 \pi}{\lambda}$ е пространствената честота.
$c$ се нарича фазова скорост, а $2 \pi (x - c t)$ - фаза.
Ако фиксираме $x$, то менейки $t$ получаваме прериодично движение нагоре-надолу.
Ако в $x$ има гребен, то следващият ще настъпи, когато към аргумента на $\cos$ добавим $2 \pi$.
Но
\begin{equation*}
  \frac{2 \pi}{\lambda}(x - c t) + 2 \pi = \frac{2 \pi}{\lambda}(x - c t + \lambda) = \frac{2 \pi}{\lambda}(x - c (t + \frac{\lambda}{c}))
\end{equation*}
Това води до $T = \frac{\lambda}{c}$ - период между двата гребена за фиксираната позиция.
$\nu = \frac{1}{T}$ се нарича циклична честота, а $\omega = 2 \pi \nu = k c$ - радиална честота.
Така може да сведем уравнението на вълната до:
\begin{equation}
  z(x,t) = A \cos (k x - \omega t)
\end{equation}
Гребените се предвижват по следния начин с времето:
\begin{equation*}
  k \xi - \omega t = 2 n \pi, \quad \xi = \frac{\omega}{k} t + \frac{2 n \pi}{k}
\end{equation*}
Така фазовата скорост $c$ се оказва точно скоростта с която се предвижва вълната в пространството за даден период от време.
Ако $c$ зависи от вълновото число $k$ (или еквивалентно от дължината на вълната $\lambda$), то наричаме повърхностните вълни дисперсни.


\subsection{Групова скорост}
Нека съберем синусоидални вълни с еднакви амплитуди и близки възлови числа. Тогава:
\begin{align*}
  z(x, t) = A \cos (k_1 x - \omega_1 t) + A \cos (\delta k_2 x - \omega_2 t) = 2 A \cos (\frac{\Delta k}{2} x - \frac{\Delta \omega}{2} t) \cos (k x - \omega t)
\end{align*}
Въвели сме $\Delta k = k_2 - k_1$, $\Delta \omega = \omega_2 - \omega_1$, $k = \frac {k_1 + k_2}{2}$, $\omega = \frac{\omega_1 + \omega_2}{2}$
Да разгледаме вълните, които в пространствен участък имат най-голяма амплитуда, т.е.:
\begin{align*}
  &\cos (\frac{\Delta k}{2} \xi - \frac{\Delta \omega}{2} t) \approx 1 \\
\t\t&\frac{\Delta k}{2} \xi - \frac{\Delta \omega}{2} t = n \pi
\end{align*}
Скоростта с която се движат точките $\xi$ на максимална амплитуда е точно $\frac{\Delta \omega}{\Delta k}$.
Това води до понятието групова скорост - $c_g = \dv{\omega}{k}$.
Обвивката на вълните с максимална амплитуда в даден момент се движи като вълна със скорост $c_g$.
Aко вълните не са дисперсни, то имаме, че:
\begin{equation*}
  c_g = \dv{\omega}{k} = \dv{c k}{k} = c
\end{equation*}
Тоест съвпада със фазовата скорост на отделните вълни. Пример за това са звуковите вълни.
Ако пък разглеждаме вълни около повърхността, то може да се покаже, че $\omega^2 = g k$, тогава:
\begin{equation*}
  c_g = \dv{\omega}{k} = \dv{\sqrt{g k}}{k} = \frac{\sqrt{g}}{2 \sqrt{k}} = \frac{c}{2}
\end{equation*}
Груповата скорост се оказва два пъти по-малка от тази на отделните вълни.
Изразявайки $\omega, k$ по дефинициите им, то достигаме до формулата на Рейли:
\begin{equation}
  c_g = \dv{\omega}{k} = \dv{\frac{2 \pi c}{\lambda}}{\frac{2 \pi}{\lambda}} = \frac{\lambda \dd c - c \dd \lambda}{- \dd \lambda} = c - \lambda \dv{c}{\lambda}
\end{equation}


Ако хвърлим камъче във водата се образуват вълни.
В момент скоро след удара, движението на водата е трудно определимо.
След време се образуват "типични" вълни с гребени и падини, но с меняща се дължина на вълната.
Вече $A, \omega, k$ ще са функции на времето и пространството.
Да разгледаме вълнов пакет описан чрез:
\begin{equation*}
  z(x, t) = A(x, t) \cos \theta(x, t)
\end{equation*}
Тук с $\theta (x, t)$ сме отбелязали локалната фаза, т.е. за фиксирани $k$ и $\omega$ - $\theta (x, t) = k x -  \omega t$.
При бавно променящ се пакет може да дефинираме $k$ и $\omega$ като производни на фазата по пространство/време:
\begin{equation*}
  k(x, t) = \pdv{\theta(x, t)}{x}, \quad  \omega(x, t) = -\pdv{\theta(x, t)}{t}
\end{equation*}
Допускайки достатъчна гладкост на $\theta$, то веднага достигаме до
\begin{equation*}
  \pdv{k}{t} + \pdv{\omega}{x} = 0
\end{equation*}
Нека сега имаме дисперсни вълни и $\omega = \omega (k)$.
Тогава използвайки диференциране на сложна функция:
\begin{equation}
  \pdv{k}{t} + c_g\pdv{k}{x} = 0
\end{equation}
Това е като материална производна, но при предвижване със скорост $c_g$.
Обаче решението на уравнението лесно се вижда, че е $f(x - c_g(k) t)$, като $f$ се определя по начални условия.
Тогава $k = const \iff x - c_g t = const$ ($c_g$ в такъв случай също би било константа).
Така всяка дължина на вълната е свързана с по една права (от деф. на $k$).
Наблюдавайки точка, движеща се с групова скорост $c_g$, то ще се наблюдава вълна с еднаква дължина.
Това е кинематичният смисъл на груповата скорост.

\subsection{Поток на енергията}
С вълновото движение се предава енергия от един обем на друг, т.е. енергията се изменя по пространството и по времето.
Нека имаме вълна движеща се отляво надясно. Тя ще има потенциал от вида:
\begin{align*}
  \numberthis
\t\t&\Phi(x, z, t) = \frac{a g}{\omega} e^{k z} \sin (kx - \omega t) \\
\t\t&u = \pdv{\Phi}{x} = \frac{a g k}{\omega} e^{k z} \cos (kx - \omega t) = a \omega e^{k z} \cos (kx - \omega t)
\end{align*}
Тук използвахме $\omega^2 = g k$.
Така трябва да пресметнем енергийния поток през равнина перпендикулярна на x.
Трабва да разгледаме само действието на $p$ за някакъв период $\tau$.
Да вземем някаква малка ивица по $z$ - $\Delta z$ и такава с дължина 1 по $y$.
Интересуваме се от работата, извършена от налягането върху частта от равнината, заключена между тях за време $\Delta t$.
За достатъчно малки $\Delta z$, може да приближим стойността на $p$, с тази в някоя точка от правоъгълника, например центъра.
За преместването използваме скоростта на вълната $u$, отново оценена там.
Така получаваме формула за работата от вида:
\begin{equation*}
  1 \Delta z p(\xi, \zeta, t) u(\xi, \zeta, t) \Delta t
\end{equation*}
След граничен преход получаваме, че
\begin{equation}
  A = \int_{-\infty}^0 \int_{0}^{\tau} p u \, \dd z \dd t
\end{equation}
Но $\frac{p - p_0}{\rho} = -\pdv{\Phi}{t} - g z = a g e^{k z} \cos (kx - \omega t) - g z$.
Сега може да изразим:
\begin{equation}
  p u = a^2 g \omega \rho e^{2 k z} \cos^2 (kx - \omega t) + (p_0 - \rho g z) a \omega e^{k z} \cos (k x - \omega t)
\end{equation}
Ако перпендикулярната равнина, която разглеждаме е точно $Oyz$, то по нея $x = 0$. Използвайки четността на $\cos$, както и
$\int_0^\tau \cos^2 \omega t \, \dd t = \frac{\tau}{2}$ и $\int_0^\tau \cos \omega t \, \dd t = 0$, то:
\begin{align*}
  \numberthis
\t\t&A =
\t\t\int_{-\infty}^0 \int_{0}^{\tau} p u \, \dd z \dd t =
\t\t\int_{-\infty}^0 \int_{0}^{\tau} a^2 g \omega \rho e^{2 k z} \cos^2 \omega t + (p_0 - \rho g z) a \omega e^{k z} \cos \omega t \, \dd z \dd t = \\
\t\t&\int_{-\infty}^0 \left( a^2 g \omega \rho e^{2 k z} \int_{0}^{\tau} \cos^2 \omega t \, \dd t + (p_0 - \rho g z) a \omega e^{k z} \int_{0}^{\tau} \cos \omega t \, \dd t \right) \, \dd z = \\
\t\t&\int_{-\infty}^0 a^2 g \omega \rho e^{2 k z} \frac{\tau}{2} \dd z =
\t\t\frac{1}{2}\frac{\tau}{2} \int_{z=-\infty}^{z=0} \frac{a^2 g^2 \rho}{\omega} e^{2 k z} \dd (2 k z) = \frac{a^2 g^2 \rho \tau}{4 \omega}
\end{align*}
Работата за единица време е $A_1 = \frac{a^2 g^2 \rho}{4 \omega}$. Тя може да бъде разписана така:
\begin{equation}
  A_1 = \frac{\rho g a^2}{4}\frac{g}{\omega} = \frac{\rho g a^2}{4}\sqrt{\frac{g}{k}}=\frac{\rho g a^2}{2}\frac{c}{2} = E c_g
\end{equation}
Така пълната енергия на вълните се пренася със скорост, равна на груповата.
