\setcounter{equation}{0}
\section{Хидродинамична неустойчивост}
Общата идея на понятието устойчивост е при малки смущения на някакво първоначално/равновесно състояние, с времето да не настъпят големи изменения спрямо него.
В хидродинамиката неустойчивостта се наблюдава под формата на хаотични, привидно произволни изменения в потока, наречени турбуленция.
Идеята на изследването на неустойчивостта е да се предвиди настъпването на промяна от ламинарен към турбулентен поток.


\subsection{Метод на нормалните моди}
Линейният анализ на устойчивостта допуска съществуването на синусоидални смущения на някакво базово/първоначално/равновесно състояние.
Например поток, успореден на $x$, но менящ се с $z$. Към този поток прилагаме смущение от вида:
\begin{equation}
  u(x, y, z, t) = \hat{u}(z) e^{ikx + imy + \sigma t} = \hat{u}(z) e^{i \norm{mathbf{K}}(\mathbf{e}_K \cdot \mathbf{x} - ct)}
\end{equation}
Физически величини се получават от реалната част.
$\hat{u}(z)$ е комплексна амплитуда, $\mathbf{K} = (k, m, 0)$ е вълновото число на смущението, $\mathbf{e}_K=\frac{\mathbf{K}}{\norm{\mathbf{K}}}$.
$\mathbf{x} = (x, y, z)$, $c$ - комплексна фазова скорост, $\sigma$ - времеви темп на растежа.
В случая в експонентата участват $x, y, t$, защото коефициентите в диференциалните уравнения, определящи смущенията са не зависят от тях.
Ако допуснем, че потокът е неограничен по $x$ и $y$, то $k$ и $m$ ще трябва да са реални.
Така ще трябва $\sigma, c$ да са комплексни, за да са изпълнени гранични условия за $x, y$ в безкрайност.

Ако $Re(\sigma)$ или $Im(c)$ са положителни за някоя стойност на вълновото число, то системата е неустойчива към неини смущения.
\begin{itemize}
  \item $Re(\sigma) < 0$ или $Im(c) < 0$ влекат устойчив поток.
\t\t\item $Re(\sigma) = 0$ или $Im(c) = 0$ влекат неутрално устойчив поток.
\t\t\item $Re(\sigma) > 0$ или $Im(c) > 0$ влекат неустойчив поток.
\end{itemize}
При метода на нормалните моди, всяка мода е независима и се разглежда отделно.
Гранично състояние - когато сме на прага на устойчивост/неустойчивост, т.е. $Re(\sigma) = Im(c) = 0$.
Ако $Im(\sigma) \neq 0$ или $Re(c) \neq 0$, то неустойчивостта се проявява като блягащи осцилации с растяща амплитуда.
В другия случай, при неустойчивост се наблюдават вторични потоци.


\subsection{Теорема на Скуайър и уравнение на Ор-Зомерфелд}
Разглеждаме хомогенен (т.е. $\rho = const$) вискозен флуид в 3 пространствени измерения.
Нека имаме основен поток насочен по $x$, като се изменя по $y$, тъй че $\mathbf{V}=(U(y), 0, 0)$.
Разбиваме общия поток на сума от поток и пертурбация.
\begin{equation*}
  
\end{equation*}
Тогава от уравненията на Навие-Стокс
\begin{align*}
  &\pdv{u}{t} + (U + u) \pdv{U + u}{x} + v \pdv{U + u}{y} + w \pdv{U + u}{z} = - \pdv{P + p}{x} + \nu \laplace (U + u) \\
\t\t&-\pdv{P}{x} + \nu \laplace U = 0
\end{align*}
След изваждане на второто от първото и пренебрегване на нелинейни спрямо $U$ членове
\begin{align*}
  &\pdv{u}{t} + U \pdv{u}{x} + v \pdv{U}{y} = - \pdv{p}{x} + \nu \laplace (u) \\
\end{align*}
Аналогично може да изведем за другите две компоненти:
\begin{align*}
  &\pdv{v}{t} + U \pdv{v}{x} = - \pdv{p}{y} + \nu \laplace (v) \\
\t\t&\pdv{w}{t} + U \pdv{w}{x} = - \pdv{p}{z} + \nu \laplace (w) \\
\t\t&\pdv{u}{x} + \pdv{v}{y} + \pdv{w}{z} = 0
\end{align*}
Уравненията са линейни и коефициентите им зависят само от $y$, откъдето решенията ще са от вида:
\begin{align*}
  &\mathbf{u} = \hat{\mathbf{u}(y)} e^{i (k x + m z - k c t)} \\
\t\t&p = \hat{p}(y) e^{i (k x + m z - k c t)}
\end{align*}
Компонентите на вълновото число $k$ и $m$ трябва да са реални, докато фазовата скорост $c = c_r + i c_i$.
Това води до аналогичния запис:
\begin{align*}
  \numberthis
\t\t&\mathbf{u} = \hat{\mathbf{u}}(y) e^{k c_i t} e^{i (k x + m z - k c_r t)} \\
\t\t&p = \hat{p}(y) e^{k c_i t} e^{i (k x + m z - k c_r t)}
\end{align*}

Ако фиксираме $x, z$, то пертурбацията извършва осцилации с честота $k c_r$, като имаме устойчивост за $c_i < 0$ и неустойчивост за $c_i > 0$.
Сега остава да заместим в уравненията.
Да припомним, че $\mathrm{Re}$ е числото на Рейнолдс и се дефинира като $\mathrm{Re} = \frac{U_0 L}{\nu}$ за някакви характеристични скорост $U_0$ и дължина $L$.
Достигаме до:
\begin{align*}
  \numberthis
\t\t&i k (U - c)\hat{u} + \hat{v}\dv{U}{y} = -i k p + \frac{U_0 L}{\mathrm{Re}}\left[\dv[2]{\hat{u}}{y} - (k^2 + m^2)\hat{u}\right] \\
\t\t&i k (U - c)\hat{v} = -\dv{\hat{p}}{y} + \frac{U_0 L}{\mathrm{Re}}\left[\dv[2]{\hat{v}}{y} - (k^2 + m^2)\hat{v}\right] \\
\t\t&i k (U - c)\hat{w} = -i m p + \frac{U_0 L}{\mathrm{Re}}\left[\dv[2]{\hat{w}}{y} - (k^2 + m^2)\hat{w}\right] \\
\t\t&i k \hat{u} + \dv{\hat{v}}{y} + i m \hat{w} = 0
\end{align*}

\begin{theorem}[Теорема на Скуайър]
  За всяко тримерно неустойчиво смущение съществува по-неустойчиво двумерно смущение.
\end{theorem}

\begin{proof}
  Прилагаме тъй наречената трансформация на Скуайър:
\t\t\begin{align*}
\t\t  &\bar{k} = \sqrt{k^2 + m^2}, \quad \bar{k}\bar{u} = k \hat{u} + m \hat{w} \\
\t\t  &\frac{\bar{p}}{\bar{k}}=\frac{\hat{p}}{k}, \quad \bar{k}\bar{\mathrm{Re}} = k \mathrm{Re} \\
\t\t  &\bar{c} = c, \quad \bar{v} = \hat{v}
\t\t\end{align*}
\t\tКогато $m \neq 0$, то $\bar{\mathrm{Re}} < \mathrm{Re}$. Събираме първото и третото от по-рано получените уравнения:
\t\t\begin{align*}
\t\t  \numberthis
\t\t  &i \bar{k} (U - c) \bar{u} + \bar{v}\dv{U}{y} = -i \bar{k} \bar{p} + \frac{U_0 L}{\bar{\mathrm{Re}}}\left[\dv[2]{\bar{u}}{y} - \bar{k}^2\bar{u}\right] \\
\t\t  &i \bar{k} (U - c) \bar{v} = -\dv{\hat{p}}{y} + \frac{U_0 L}{\bar{\mathrm{Re}}}\left[\dv[2]{\bar{v}}{y} - \bar{k}\bar{v}\right] \\
\t\t  &i \bar{k} \bar{u} + \dv{\bar{v}}{y} = 0
\t\t\end{align*}
\t\tТъй като $\bar{\mathrm{Re}} < \mathrm{Re}$, то за този двумерен поток критичното число на Рейнолдс, където започва неустойчивостта, ще бъде достигнато преди това за тримерния.
\end{proof}

Интерпретацията е следната. Тримерното смущение е вълна, разпространяваща се косо на равновесния поток.
При правилна промяна на координатната система ($x$ да сочи в тази посока), то само по новата ос $x$ равновесния поток въздейства на смущението.
Така е достатъчно да разглеждаме $m = \hat{\omega} = 0$.

В новия двумерен поток е вече може да дефинираме фунцкия на тока $\psi(x,y,t)$.
След прилагане на метода на нормалните моди:
\begin{align*}
  \numberthis
\t\t&u = \hat{u}(y) e^{i k (x - c t)} \\
\t\t&v = \hat{v}(y) e^{i k (x - c t)} \\
\t\t&\psi = \hat{\psi}(y) e^{i k (x - c t)} \\
\end{align*}

Лесно може да видим, че $\hat{u} = \pdv{\hat{\psi}}{y}$ и $\hat{v} = - i k \hat{\psi}$.
Сега от първото уравнение за двумерния поток може да изразим $p$ и да заместим във второто.
Използваме и тези формули и достигаме до ОДУ от четвърти ред спрямо $\hat{\psi}$, което е известно като уравнение на Ор-Зомерфелд:
\begin{equation}
  (U - c) (\dv[2]{\hat{\psi}}{y} - k^2 \hat{\psi}) - \dv[2]{U}{y}\hat{\psi} = \frac{\nu}{i k} (\dv[4]{\hat{\psi}}{y} - 2 k^2 \dv[2]{\hat{\psi}}{y} + k^4 \hat{\psi})
\end{equation}
За гранични условия имаме 2 варианта:
\begin{itemize}
  \item Ограничени между някакви $y_1$ и $y_2$ несмутени потоци:
\t\t\begin{equation*}
\t\t  \hat{\psi}\vert_{y=y_1} = \hat{\psi}\vert_{y=y_2} = \dv{\hat{\psi}}{y}\vert_{y=y_1} = \dv{\hat{\psi}}{y}\vert_{y=y_2} = 0
\t\t\end{equation*}
\t\t\item Неограничени несмутени потоци. За тях трябва приложените смущения да изчезват, отдалечавайки се от началото на координатната система:
\t\t\begin{equation*}
\t\t  \hat{\psi}\vert_{y=\infty} = \dv{\hat{\psi}}{y}\vert_{y=\infty} = 0
\t\t\end{equation*}
\end{itemize}
Аналитични решения или тяхни свойства са намерени само за много прости потоци.
Това практически е уравнение за вихъра.
