% Meta
\documentclass[bulgarian, 12pt]{article}
\usepackage[
	a4paper, 
	includeheadfoot, 
	margin = 1.5 cm]
{geometry}

% Hyperlinks
% \usepackage[
% 	unicode=true, 
% 	colorlinks=true, 
% 	linkcolor=black, 
% 	urlcolor=black]
% {hyperref}

% Fonts
\usepackage[T2A]{fontenc}
\usepackage[utf8]{inputenc}
\usepackage[bulgarian]{babel}
\usepackage{bm}
% ISO-Math (only for XeLaTeX and LuaLaTex)
%\usepackage[math-style=ISO]{unicode-math}

% Citing
% IMPORTANT! USE 'babel=true' to be able to use csquotes with a multitude of languages
% By default you can use only ~10 languages
\usepackage[babel=true]{csquotes}

% Indent first line in paragraph
\usepackage{indentfirst}

% Place tags on the left
\usepackage[leqno]{amsmath}

% Better math
\usepackage{amsbsy}
\usepackage{amssymb}
\usepackage{mathtools}
\usepackage{comment}
\usepackage{mathptmx}
\usepackage[makeroom]{cancel}

% Create math pictures
\usepackage{tikz}
\usepackage{enumitem}

% Better theorems
\usepackage{amsthm}

% Derivative notations
\usepackage{physics}
%\usepackage{derivative}

%%%%%%%%%%%%%%%%%%%%%%%%%%%%%%%%%%%%%%%

% Lapacian delta
%\newcommand{\laplace}{\increment}
\newcommand{\laplace}{∆}
%\fontsize{16pt}{20pt}\selectfont
% Bolded cyrilic text
\renewcommand{\sfdefault}{cmss}
\renewcommand{\rmdefault}{cmr}
\renewcommand{\ttdefault}{cmt}

% Roman numerals for sections
\renewcommand{\thesection}{\Roman{section}} 
%\renewcommand{\thesubsection}{\thesection.\Roman{subsection}}
% Sectioning titles
\newtheorem{definition}{Дефиниция}[section]
\newtheorem{problem}{Задача}
\newtheorem{theorem}{Теорема}
\newtheorem*{theorem*}{Теорема}
\newtheorem{lemma}{Лема}
\newtheorem*{solution*}{Решение}

% Numbering of equations
\newcommand\numberthis{\addtocounter{equation}{1}\tag{\theequation}}

% Space between lines in array for fractions
\renewcommand{\arraystretch}{1.5}

\title{Избрани въпроси от хидродинамиката}

\author{Калоян Стоилов}

\begin{document}
\maketitle

\section{Изменение на количеството на движение}
Количеството движение или още - импулс в механиката на твърди тела се нарича $\mathbf{K} = m\mathbf{v}$ (често се бележи с $p$, но при нас това е налягането).
Вторият закон на Нютон гласи, че скоростта на изменение на импулса е равно на равнодействащата сила $\mathbf{F} = \dv{\mathbf{K}}{t} = \dv{m\mathbf{v}}{t}= m \dot{\mathbf{v}}$. 
Законът е в сила за тела, непроменящи масата си.

\subsection{Масови и повърхностни сили}
Нека $\tau$ е обем от флуид с маса $M$. 
Масовата сила е действащата на флуида в обема сила, която не зависи от взаимодействието с други части на флуида. 
Нека $\mathbf{F}_M$ е главния вектор на силите (т.е. равнодействащата сила), действащи на флуида във $\tau$.
Средна масова сила, действаща върху маса $M$ се нарича $F_{avg} = \frac{\mathbf{F}_M}{M}$.
Масова сила $\mathbf{F}$ в точка $B$, наричаме
\begin{equation}
	\mathbf{F} = \lim_{\tau \to \{B\}} F_{avg} = \lim_{\tau \to \{B\}} \frac{\mathbf{F}_M}{M}
\end{equation}
Ако знаем $\mathbf{F}$ в коя да е точка от $\tau$, то може да получим $\mathbf{F}_M$.
Наистина, нека $\Delta \tau$ е обем с маса $\Delta m = \rho \Delta \tau$, на който действа $\mathbf{F}_avg \Delta m$.
Разбивайки $\tau$ на такива обеми, може да съберем всички такива сили и след граничен преход получаваме:
\begin{equation}
	\mathbf{F}_M = \iiint\limits_{\tau} \mathbf{F} \dd m = \iiint\limits_{\tau} \rho \mathbf{F} \dd \tau
\end{equation}
Нека обемът е ограничен от повърхнина $S$. Флуидът извън $\tau$, действа на този във $\tau$ през $S$ чрез повърхностни сили.
Нека приближим част от повърхнината с равнинна част $\Delta S$ с нормала $\mathbf{n}$, a главния вектор на силите, действащи ѝ e $\Delta F_S^n$.
Средното напрежение, действащо на площта е $\mathbf{t}_{avg}^n = \frac{\Delta \mathbf{F}_S^n}{\Delta S}$. 
Напрежение $\mathbf{t}^n$ на повърхностни сили, действащи в точка $B$, наричаме
\begin{equation}
	\mathbf{t} = \lim_{\Delta S \to \{B\}} \mathbf{t}_{avg}^n = \lim_{\Delta S \to \{B\}} \frac{\Delta F_S^n}{\Delta S}
\end{equation}
Отново сумираме всички такива сили за $S$ и след граничен преход главният вектор на повърхностните сили е:
\begin{equation}
	\mathbf{F}_S = \iint\limits_{S} \dd \mathbf{F}_S^n = \iint\limits_{S} \mathbf{t}^n \dd S
\end{equation}

\subsection{Интегрална форма на закона за изменение на количеството на движение}
В малък обем $\Delta \tau$ с маса $\rho \Delta \tau$ ще имаме импулс $\Delta \mathbf{K} = \rho \mathbf{v} \Delta \tau$.
Така количеството движение на флуида ще бъде
\begin{equation}
	\mathbf{K} = \iiint\limits_{\tau} \dd \mathbf{K} = \iiint\limits_{\tau} \rho \mathbf{v} \dd \tau
\end{equation}
Тъй като силите, действащи на $\tau$ или са масови, или повърхностни, то вторият закон на Нютон придобива вида:
\begin{equation}
	\dv{t}\iiint\limits_{\tau} \rho \mathbf{v} \dd \tau = \iiint\limits_{\tau} \rho \mathbf{F} \dd \tau + \iint\limits_{S} \mathbf{t}^n \dd S
\end{equation}
Не бива да забравяме, че и самият обем $\tau$ се мени с времето. Тогава ще имаме 
\begin{equation}
	\dv{t}\iiint\limits_{\tau} \rho \mathbf{v} \dd \tau = \iiint\limits_{\tau} \dv{\rho \mathbf{v}}{t} + \rho \mathbf{v} \div \mathbf{v} \dd \tau 
\end{equation}
Така получаваме интегралната форма на закона за изменение на количеството движение
\begin{equation}
	\iiint\limits_{\tau} \dv{\rho \mathbf{v}}{t} + \rho \mathbf{v} \div \mathbf{v} - \rho \mathbf{F} \dd \tau = \iint\limits_{S} \mathbf{t}^n \dd S
\end{equation}

\subsection{Изменение на интегрално количество}
Нека $Q$ бъде някаква величина - скаларна или векторна, която е дефинирана поточково в обем $\tau$. 
Тогава изменението по времето на общата величина за обема ще бъде 
\begin{equation}
	\dv{t}\iiint\limits_{\tau} Q \dd \tau
\end{equation}
Тъй като говорим за флуиди и самият обем се мени с времето. Да разгледаме $\tau(t + \Delta t) - \tau (t)$.
За достатъчно малко време и малка площ $\Delta S$ по границата $S(t)$, може да разглеждаме че се движи със скорост $\mathbf{v} \cdot \mathbf{n}$ към нова повърхнина $S(t + \Delta t)$.
Така изменението на обема над тази площ ще може да се пресметне като обем на прав криволинеен цилиндър 
\begin{equation}
	\Delta \tau = h \Delta S = (\mathbf{v} \cdot \mathbf{n}) \Delta t \Delta S 
\end{equation}
След граничен преход и изразявайки обема чрез интеграл по елементарни обеми получаваме:
\begin{equation}
	\frac{\iiint\limits_{\tau(t + \Delta t) - \tau (t)} \dd \tau}{\Delta t} = \iint\limits_{S} \mathbf{v} \cdot \mathbf{n} \dd S 
\end{equation}
Сега може да получим аналог на формулата за диференциране на Лайбниц
\begin{equation}
	\dv{t}\iiint\limits_{\tau} Q \dd \tau = \iiint\limits_{\tau} \pdv{Q}{t} \dd \tau + \iint\limits_{S} Q (\mathbf{v} \cdot \mathbf{n}) \dd S
\end{equation}
Може да забележим, че ако $Q$ е скаларна величина, то $Q(\mathbf{v} \cdot \mathbf{n}) = (Q \mathbf{v}) \cdot \mathbf{n}$. 
Използвайки теоремата на Гаус-Остроградски, то
\begin{equation}
	\dv{t}\iiint\limits_{\tau} Q \dd \tau = \iiint\limits_{\tau} \pdv{Q}{t} \dd \tau + \iiint\limits_{\tau} \div (Q \mathbf{v}) \dd \tau
\end{equation}
Лесно може да се провери, че
\begin{equation}
	\div (Q \mathbf{v}) = \grad Q \cdot \mathbf{v} + Q \div \mathbf{v} = \grad Q \cdot \dot{\mathbf{x}} + Q \div \mathbf{v}
\end{equation}
Остава да забележим, че
\begin{equation}
	\pdv{Q}{t} + \grad Q \cdot \dot{\mathbf{x}} = \dv{Q}{t}
\end{equation}
След използване на линейността на интеграла получаваме
\begin{equation}
	\dv{t}\iiint\limits_{\tau} Q \dd \tau = \iiint\limits_{\tau} \dv{Q}{t} + Q \div \mathbf{v} \dd \tau
\end{equation}
Ако $Q$ е векторна величина, то може да го разгледаме покомпонентно и пак получаваме същата формула.

\subsection{Формула на Коши}
Нека $\tau$ бъде триъгълна пирамида с прав тристенен ъгъл при върха си - началото на координатната система.
Тогава може да се опише като съвкупност от 4 повъхнини:
\begin{enumerate}
	\item $S_x$ е стената перпендикулярна на оста $x$.
	\item $S_y$ е стената перпендикулярна на оста $y$.
	\item $S_z$ е стената перпендикулярна на оста $z$.
	\item $S_n$ е стената срещу тристенният ъгъл на координатната система.
\end{enumerate}
Тогава $\mathbf{t}^{-x}$, $\mathbf{t}^{-y}$, $\mathbf{t}^{-z}$ ще са напреженията по съответните първи три стени.
Нека $\mathbf{t}^{n}$ бъде по четвъртата. Така се достига до формулата на Коши
\begin{equation}
	\iiint\limits_{\tau} \dv{\rho \mathbf{v}}{t} + \rho \mathbf{v} \div \mathbf{v} - \rho \mathbf{F} \dd \tau =
	\iint\limits_{S_x} \mathbf{t}^{-x} \dd S + \iint\limits_{S_y} \mathbf{t}^{-y} \dd S + \iint\limits_{S_z} \mathbf{t}^{-z} \dd S + \iint\limits_{S_n} \mathbf{t}^{n} \dd S
\end{equation}


\section{Безразмерни течения}
Ще разгледаме вискозни течения с непроменлива динамична вискозност $\mu$.
Същото предполагаме и за масовите сили $\mathbf{g}$.
Експериментални изследвания върху течения с модели/макети могат да служат за качествено/количествено характеризиране на по-големи обекти, които на практика могат да се ползват (напр. кораби, самолети).
За тази цел се използва обезразмеряване.

\subsection{Безразмерен запис на уравнения на течения}
Нека разгледаме системата от уравнения на Навие-Стокс и уравнението на непрекъснатостта
\begin{gather}
	\pdv{\mathbf{v}}{t} + \mathbf{v} \cdot \grad \mathbf{v} = \mathbf{g} - \frac{1}{\rho}\grad p + \nu \laplacian \mathbf{v} \\\nonumber
	\div \mathbf{v} = 0
\end{gather}
Разглеждаме тяло с характерна дължина $l$. Правим смяна на координатите, като искаме да разпишем уравненията в следната система
\begin{equation}
	\xi = \frac{x}{l},\, \eta = \frac{y}{l},\, \zeta = \frac{z}{l},\, \tau = \frac{t}{\frac{l}{\nu}}
\end{equation}
Въвеждаме безразмерни функции
\begin{equation}
	\mathbf{u} = \frac{l}{\nu} \mathbf{v},\, \Pi = \frac{l^2}{\nu^2} \frac{p}{\rho},\, \bm{\gamma} = \frac{l^3}{\nu^2} \mathbf{g}
\end{equation}
Трябва да ги запишем като функции на новите координати. За да сведем уравненията използваме, че
\begin{align}
	&\pdv{\mathbf{v}}{t} = \pdv{(\frac{\nu}{l}\mathbf{u})}{\tau}\dv{\tau}{t} = \frac{\nu^2}{l^3} \pdv{\mathbf{u}}{\tau} \\
	&\mathbf{v} \cdot \grad \mathbf{v} = 
		\frac{\nu}{l} \mathbf{u} \cdot \grad \frac{\nu}{l} \mathbf{u} = 
		\frac{\nu^2}{l^2} \mathbf{u} \cdot (\pdv{\mathbf{u}}{\xi}\dv{\xi}{x} + \pdv{\mathbf{u}}{\eta}\dv{\eta}{y} + \pdv{\mathbf{u}}{\zeta}\dv{\zeta}{z}) =
		\frac{\nu^2}{l^3} \mathbf{u} \cdot (\pdv{\mathbf{u}}{\xi} + \pdv{\mathbf{u}}{\eta} + \pdv{\mathbf{u}}{\zeta}) \\
	& \div \mathbf{v} = 0 \iff \pdv{u_x}{\xi} + \pdv{u_y}{\eta} + \pdv{u_z}{\zeta} = 0 \\
	& \nu \laplacian \mathbf{v} = 
		\nu \div \frac{\nu}{l} (\pdv{\mathbf{u}}{\xi}\dv{\xi}{x} + \pdv{\mathbf{u}}{\eta}\dv{\eta}{y} + \pdv{\mathbf{u}}{\zeta}\dv{\zeta}{z}) = 
		\frac{\nu^2}{l^2} (\pdv[2]{\mathbf{u}}{\xi}\dv{\xi}{x} + \pdv[2]{\mathbf{u}}{\eta}\dv{\eta}{y} + \pdv[2]{\mathbf{u}}{\zeta}\dv{\zeta}{z}) =
		\frac{\nu^2}{l^3} (\pdv[2]{\mathbf{u}}{\xi} + \pdv[2]{\mathbf{u}}{\eta} + \pdv[2]{\mathbf{u}}{\zeta}) \\
	&\frac{1}{\rho}\grad p = 
		\grad \frac{p}{\rho} =
	 	\frac{\nu^2}{l^2} (\pdv{\Pi}{\xi}\dv{\xi}{x} + \pdv{\Pi}{\eta}\dv{\eta}{y} + \pdv{\Pi}{\zeta}\dv{\zeta}{z}) =
		\frac{\nu^2}{l^3} (\pdv{\Pi}{\xi} + \pdv{\Pi}{\eta} + \pdv{\Pi}{\zeta})
\end{align}
Съкращаваме и получаваме системата
\begin{gather}
	\dv{\mathbf{u}}{\tau} = \bm{\gamma} - \grad \Pi + \laplacian \mathbf{u} \\\nonumber
	\div \mathbf{u} = 0
\end{gather}
Тук операторите са спрямо новите ни променливи $\xi, \eta, \zeta$, като вече единицата за дължина е характерната дължина на тялото, т.е. $l$.

\subsection{Подобни координати}
Нека имаме две подобни тела със съответни характерни дължини $l_1$, $l_2$ - те ще определят линейния мащаб за двете задачи. 
Изразяваме съответно течения с кинематични вискозности $\nu_i = \frac{\mu_i}{\rho_i}$ в координати $x_i,y_i,z_i,t_i, \quad i=1,2$.
Да забележим, че $[\nu_i]=\frac{L^2}{T}$, а $[l_i]=L$.
Така за мащаб по времето може да вземем $\frac{l_i^2}{\nu_i},\, i=1,2$. 
За обезразмерени уравнения въвеждаме координати
\begin{equation}
	\xi_i = \frac{x_i}{l_i},\, \eta_i = \frac{y_i}{l_i},\, \zeta_i = \frac{z_i}{l_i},\, \tau_i = \frac{t_i}{\frac{l_i^2}{\nu_i}}, \quad i=1,2
\end{equation}
Подобни координати на двете течения наричаме тези, за които всички двойки безразмерни величини съвпадат.
След тези преобразузавания и двете безразмерни тела имат характерни дължини $1$ и са геометрически еднакви. 
\subsection{Подобие при вискозни течения}
Иска ни се с едно течение да оприличим друго - както например имаме геометрично подобие на фигури и сме извели някакво свойство/количество за една от тях, лесно може да го получим за другата.
Ще казваме, че две течения са подобни, ако са около подобни тела и стойностите на техните хидромеханични величини в подобни координати са еднакви с точност до константен множител (не задължително еднакъв за различните величини). 
И тъй нека имаме две течения със съответни величини 
\begin{equation}
	l_i, \mathbf{v}_i, \mathbf{g}_i, \nu_i, \frac{p_i}{\rho_i}, \quad i=1,2
\end{equation}
Теченията имат и безразмерни уравнения и нека разгледаме задачата за обтичане по тяло. 
В безразмерни координати телата се изобразяват в "единично" тяло със същата форма.
Нека бележим границата му с $S$. 
\begin{enumerate}
	\item Трябва да са изпълнени граничните условия по границата на тялото - $\mathbf{u}_1\vert_S = \mathbf{u}_2\vert_S = \mathbf{0}$.
	\item Трябва да са изпълнени граничните условия в безкрайност - $\mathbf{u}_1\vert_\infty = \mathbf{U}_1,\, \mathbf{u}_2\vert_\infty = \mathbf{U}_2$.
\end{enumerate}
Достатъчно е да са изпълнени следните равенства за безразмерните величини - $\mathbf{u}_1 = \mathbf{u}_2,\, \Pi_1 = \Pi_2$.
За да са изпълнени е достатъчно двете безразмерни уравнения да съвпадат, както и граничните условия да съвпадат.

Очевидно граничните условия по границата на тялото са едни и същи.
За да съвпадат тези в безкрайност, то трябва
\begin{equation}
	\frac{\mathbf{V}_1 l_1}{\nu_1} = \mathbf{U}_1 = \mathbf{U}_2 = \frac{\mathbf{V}_2 l_2}{\nu_2}
\end{equation}

За да съвпадат уравненията ще трябва
\begin{equation}
	\frac{\mathbf{g}_1 l_1^3}{\nu_1^2} = \bm{\gamma}_1 = \bm{\gamma}_2 = \frac{\mathbf{g}_2 l_2^3}{\nu_2^2}
\end{equation}

Последните две уравнения са векторни. Изпълнени са точно когато съответните вектори от двете страни са колинеарни и
\begin{gather}
	\label{ReSame} \frac{\norm{\mathbf{V}_1} l_1}{\nu_1} = \frac{\norm{\mathbf{V}_2} l_2}{\nu_2}\\
	\frac{\norm{\mathbf{g}_1} l_1^3}{\nu_1^2} = \frac{\norm{\mathbf{g}_2} l_2^3}{\nu_2^2}
\end{gather}
Обикновено се взима еквивалента система уравнения - \eqref{ReSame} и 
\begin{equation}
	\label{FrSame} \frac{\norm{\mathbf{V}_1}^2}{\norm{\mathbf{g}_1} l_1} = \frac{\norm{\mathbf{V}_2}^2}{\norm{\mathbf{g}_2} l_2}
\end{equation}

\subsection{Безразмерни характерни числа}
\subsubsection{Основни безразмерни характерни числа}
Числото на Рейнолдс $Re = \frac{l v_\infty}{\nu}$ има значение само за вискозни флуиди, т.к. за идеалните $\nu = 0$, т.е. $Re = \infty$.
То представлява отношението на инерчните сили към вискозните сили.

Числото на Фруд $Fr = \frac{v_\infty}{\sqrt{l g}}$ има смисъл и за невискозни флуиди.
То представлява отношението на инерчните сили към гравитационните сили.

Така за да са подобни две течения, трябва числата им на Рейнолдс и Фруд да съвпадат.
\subsubsection{Число на Рейнолдс}
Числото на Рейнолдс $Re = \frac{l v_\infty}{\nu}$ има значение само за вискозни флуиди, т.к. за невискозни $\nu = 0$, т.е. $Re = \infty$.
То представлява отношението на инерчните сили към вискозните сили.
\subsubsection{Число на Фруд}
Числото на Фруд $Fr = \frac{v_\infty}{\sqrt{l g}}$ има смисъл и за невискозни флуиди.
То представлява отношението на инерчните сили към гравитационните сили.
\subsubsection{Число на Струхал}
Числото на Струхал $St = \frac{v_\infty}{\sqrt{l g}}$ има смисъл и за невискозни флуиди.
То представлява отношението на вихровото ускорение към адвективното ускорение.
\subsubsection{Число на Мах}
Числото на Мах $M = \frac{v_\infty}{\sqrt{l g}}$ има смисъл и за невискозни флуиди.
\subsubsection{Число на Екерт}
Числото на Екерт $Ec = \frac{v_\infty}{\sqrt{l g}}$ има смисъл и за невискозни флуиди.
\subsubsection{Число на Прантл}
Числото на Прантл $Pr = \frac{v_\infty}{\sqrt{l g}}$ има смисъл и за невискозни флуиди.
\subsubsection{Число на Вебер}
Числото на Вебер $We = \frac{v_\infty}{\sqrt{l g}}$ има смисъл и за невискозни флуиди.
\subsubsection{Число на Бонд}
Числото на Бонд $Bo = \frac{v_\infty}{\sqrt{l g}}$ има смисъл и за невискозни флуиди.
\subsubsection{Капилиарно число}
Капилиарното число $F_r = \frac{v_\infty}{\sqrt{l g}}$ има смисъл и за невискозни флуиди.
\end{document}
