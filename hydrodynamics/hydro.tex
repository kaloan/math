% Meta
\documentclass[bulgarian, 12pt]{article}
\usepackage[
	a4paper, 
	includeheadfoot, 
	margin = 1.5 cm]
{geometry}

% Hyperlinks
\usepackage[
	unicode=true, 
	colorlinks=true, 
	linkcolor=black, 
	urlcolor=black]
{hyperref}

% Fonts
\usepackage[T2A]{fontenc}
\usepackage[utf8]{inputenc}
\usepackage[bulgarian]{babel}
\usepackage{bm}
% ISO-Math (only for XeLaTeX and LuaLaTex)
%\usepackage[math-style=ISO]{unicode-math}

% Citing
% IMPORTANT! USE 'babel=true' to be able to use csquotes with a multitude of languages
% By default you can use only ~10 languages
\usepackage[babel=true]{csquotes}

% Indent first line in paragraph
\usepackage{indentfirst}

% Place tags on the left
\usepackage[leqno]{amsmath}

% Better math
\usepackage{amssymb}
\usepackage{mathtools}
\usepackage{comment}
\usepackage{mathptmx}
\usepackage[makeroom]{cancel}

% Create math pictures
\usepackage{tikz}
\usepackage{enumitem}

% Better theorems
\usepackage{amsthm}

% Derivative notations
\usepackage{physics}
\usepackage{derivative}

%%%%%%%%%%%%%%%%%%%%%%%%%%%%%%%%%%%%%%%

% Lapacian delta
%\newcommand{\laplace}{\increment}
\newcommand{\laplace}{∆}
%\fontsize{16pt}{20pt}\selectfont
% Bolded cyrilic text
\renewcommand{\sfdefault}{cmss}
\renewcommand{\rmdefault}{cmr}
\renewcommand{\ttdefault}{cmt}

% Roman numerals for sections
\renewcommand{\thesection}{\Roman{section}} 
%\renewcommand{\thesubsection}{\thesection.\Roman{subsection}}
% Sectioning titles
\newtheorem{definition}{Дефиниция}[section]
\newtheorem{problem}{Задача}
\newtheorem{theorem}{Теорема}
\newtheorem*{theorem*}{Теорема}
\newtheorem{lemma}{Лема}
\newtheorem*{solution*}{Решение}

% Numbering of equations
\newcommand\numberthis{\addtocounter{equation}{1}\tag{\theequation}}

% Space between lines in array for fractions
\renewcommand{\arraystretch}{1.5}

\title{Избрани въпроси от хидродинамиката}

\author{Калоян Стоилов}

\begin{document}
\maketitle

\section{Изменение на количеството на движение}
Количеството движение или още - импулс в механиката на твърди тела се нарича $\mathbf{K} = m\mathbf{v}$ (често се бележи с $p$, но при нас това е налягането).
Вторият закон на Нютон гласи, че скоростта на изменение на импулса е равно на равнодействащата сила $\mathbf{F} = \dv{\mathbf{K}}{t} = \dv{m\mathbf{v}}{t}= m \dot{\mathbf{v}}$. 
Законът е в сила за тела, непроменящи масата си.
\subsection{Масови и повърхностни сили}
Нека $\tau$ е обем от флуид с маса $M$. 
Масовата сила е действащата на флуида в обема сила, която не зависи от взаимодействието с други части на флуида. 
Нека $\mathbf{F}_M$ е главния вектор на силите (т.е. равнодействащата сила), действащи на флуида във $\tau$.
Средна масова сила, действаща върху маса $M$ се нарича $F_{avg} = \frac{\mathbf{F}_M}{M}$.
Масова сила $\mathbf{F}$ в точка $B$, наричаме
\begin{equation}
	\mathbf{F} = \lim_{\tau \to \{B\}} F_{avg} = \lim_{\tau \to \{B\}} \frac{\mathbf{F}_M}{M}
\end{equation}
Ако знаем $\mathbf{F}$ в коя да е точка от $\tau$, то може да получим $\mathbf{F}_M$.
Наистина, нека $\Delta \tau$ е обем с маса $\Delta m = \rho \Delta \tau$, на който действа $\mathbf{F}_avg \Delta m$.
Разбивайки $\tau$ на такива обеми, може да съберем всички такива сили и след граничен преход получаваме:
\begin{equation}
	\mathbf{F}_M = \iiint\limits_{\tau} \mathbf{F} \dd m = \iiint\limits_{\tau} \rho \mathbf{F} \dd \tau
\end{equation}
Нека обемът е ограничен от повърхнина $S$. Флуидът извън $\tau$, действа на този във $\tau$ през $S$ чрез повърхностни сили.
Нека приближим част от повърхнината с равнинна част $\Delta S$ с нормала $\mathbf{n}$, a главния вектор на силите, действащи ѝ e $\Delta F_S^n$.
Средното напрежение, действащо на площта е $\mathbf{t}_{avg}^n = \frac{\Delta \mathbf{F}_S^n}{\Delta S}$. 
Напрежение $\mathbf{t}^n$ на повърхностни сили, действащи в точка $B$, наричаме
\begin{equation}
	\mathbf{t} = \lim_{\Delta S \to \{B\}} \mathbf{t}_{avg}^n = \lim_{\Delta S \to \{B\}} \frac{\Delta F_S^n}{\Delta S}
\end{equation}
Отново сумираме всички такива сили за $S$ и след граничен преход главният вектор на повърхностните сили е:
\begin{equation}
	\mathbf{F}_S = \iint\limits_{S} \dd \mathbf{F}_S^n = \iint\limits_{S} \mathbf{t}^n \dd S
\end{equation}
\subsection{Интегрална форма на закона за изменение на количеството на движение}
В малък обем $\Delta \tau$ с маса $\rho \Delta \tau$ ще имаме импулс $\Delta \mathbf{K} = \rho \mathbf{v} \Delta \tau$.
Така количеството движение на флуида ще бъде
\begin{equation}
	\mathbf{K} = \iiint\limits_{\tau} \dd \mathbf{K} = \iiint\limits_{\tau} \rho \mathbf{v} \dd \tau
\end{equation}
Тъй като силите, действащи на $\tau$ или са масови, или повърхностни, то вторият закон на Нютон придобива вида:
\begin{equation}
	\dv{t}\iiint\limits_{\tau} \rho \mathbf{v} \dd \tau = \iiint\limits_{\tau} \rho \mathbf{F} \dd \tau + \iint\limits_{S} \mathbf{t}^n \dd S
\end{equation}
Не бива да забравяме, че и самият обем $\tau$ се мени с времето. Тогава ще имаме 
\begin{equation}
	\dv{t}\iiint\limits_{\tau} \rho \mathbf{v} \dd \tau = \iiint\limits_{\tau} \dv{\rho \mathbf{v}}{t} + \rho \mathbf{v} \div \mathbf{v} \dd \tau 
\end{equation}
Така получаваме интегралната форма на закона за изменение на количеството движение
\begin{equation}
	\iiint\limits_{\tau} \dv{\rho \mathbf{v}}{t} + \rho \mathbf{v} \div \mathbf{v} - \rho \mathbf{F} \dd \tau = \iint\limits_{S} \mathbf{t}^n \dd S
\end{equation}
\subsection{Формула на Коши}
Нека $\tau$ бъде триъгълна пирамида с прав тристенен ъгъл при върха си - началото на координатната система.
Тогава може да се опише като съвкупност от 4 повъхнини:
\begin{enumerate}
	\item $S_x$ е стената перпендикулярна на оста $x$.
	\item $S_y$ е стената перпендикулярна на оста $y$.
	\item $S_z$ е стената перпендикулярна на оста $z$.
	\item $S_n$ е стената срещу тристенният ъгъл на координатната система.
\end{enumerate}
Тогава $\mathbf{t}^{-x}$, $\mathbf{t}^{-y}$, $\mathbf{t}^{-z}$ ще са напреженията по съответните първи три стени.
Нека $\mathbf{t}^{n}$ бъде по четвъртата. Така се достига до формулата на Коши
\begin{equation}
	\iiint\limits_{\tau} \dv{\rho \mathbf{v}}{t} + \rho \mathbf{v} \div \mathbf{v} - \rho \mathbf{F} \dd \tau =
	\iint\limits_{S_x} \mathbf{t}^{-x} \dd S + \iint\limits_{S_y} \mathbf{t}^{-y} \dd S + \iint\limits_{S_z} \mathbf{t}^{-z} \dd S + \iint\limits_{S_n} \mathbf{t}^{n} \dd S
\end{equation}
\section{Подобие при вискозни течения}
\section{Безразмерни характерни числа}
\end{document}
