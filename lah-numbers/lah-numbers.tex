\documentclass[12pt]{article}
\usepackage[a4paper, includeheadfoot, margin = 1.5cm]{geometry}
\usepackage[unicode=true, colorlinks=true, linkcolor=black, urlcolor=black]{hyperref}
\usepackage[T2A]{fontenc}
\usepackage[utf8x]{inputenc}
\usepackage[bulgarian]{babel}
\usepackage{csquotes}
\usepackage{indentfirst}
\usepackage{amsmath}
\usepackage{amsthm}
\usepackage{amssymb}
\usepackage{mathtools}
\usepackage{comment}
\usepackage{mathptmx}
\usepackage{tikz}
% \usepackage{slashbox}
\usepackage{diagbox}
\usepackage{datetime}

\title{Числа на Лах}
\author{Калоян Стоилов}
\date{\formatdate{29}{4}{2021}}


\renewcommand{\sfdefault}{cmss}
\renewcommand{\rmdefault}{cmr}
\renewcommand{\ttdefault}{cmt}


\newcommand{\fallingfactorial}[1]{%
  ^{\underline{#1}}%
}
\newcommand{\risingfactorial}[1]{%
  ^{\overline{#1}}%
}
\newcommand{\genstirlingI}[3]{%
  \genfrac{[}{]}{0pt}{#1}{#2}{#3}%
}
\newcommand{\genstirlingII}[3]{%
  \genfrac{\{}{\}}{0pt}{#1}{#2}{#3}%
}

\newcommand{\stirlingI}[2]{\genstirlingI{}{#1}{#2}}
\newcommand{\dstirlingI}[2]{\genstirlingI{0}{#1}{#2}}
\newcommand{\tstirlingI}[2]{\genstirlingI{1}{#1}{#2}}
\newcommand{\stirlingII}[2]{\genstirlingII{}{#1}{#2}}
\newcommand{\dstirlingII}[2]{\genstirlingII{0}{#1}{#2}}
\newcommand{\tstirlingII}[2]{\genstirlingII{1}{#1}{#2}}


\newcommand{\genlah}[3]{%
  \genfrac{\lfloor}{\rfloor}{0pt}{#1}{#2}{#3}%
}

\begin{comment}
\newcommand{\genlah}[3]{%
  \genfrac{\rotatebox[origin=c]{180}{\rceil}}{\rotatebox[origin=c]{180}{\lceil}}{0pt}{#1}{#2}{#3}%
}
\end{comment}


\newcommand{\genover}[3]{%
  \genfrac{}{}{0pt}{#1}{#2}{#3}%
}
\newcommand{\myover}[2]{\genover{}{#1}{#2}}
\newcommand{\dover}[2]{\genover{0}{#1}{#2}}
\newcommand{\tover}[2]{\genover{1}{#1}{#2}}

\newcommand{\lah}[2]{\genstirlingI{}{#1}{#2}}
\newcommand{\dlah}[2]{\genstirlingI{0}{#1}{#2}}
\newcommand{\tlah}[2]{\genstirlingI{1}{#1}{#2}}
%\fontsize{16pt}{20pt}\selectfont

\newcommand{\lahfinal}[2]{\left\lfloor \myover{#1}{#2} \right\rfloor}




\newtheorem{theorem}{Твърдение}
\newtheorem{lemma}[theorem]{Lemma}



\begin{document}

% \begin{comment}
% \begin{titlepage}

% \begin{center}
% \begin{LARGE}

% \textbf{Числа на Лах} \\

% \end{LARGE}

% \begin{Large}

% \textbf{Калоян Стоилов}

% \end{Large}
% \end{center}
% \end{titlepage}

% \end{comment}






% \begin{center}
% \begin{Huge}


% \textbf{Числа на Лах} \\

% \end{Huge}

% \begin{Large}
% \bfseries \textit{ \slshape Калоян Стоилов}
% \end{Large}
% \end{center}

\maketitle
\begin{large}
Числата на Лах за първи път се появяват 1954г. в труд на Иво Лах(Ivo Lah), словенски математик, живял 05.09.1896-23.03.1979. Имат връзка с числата на Стирлинг. \\

Подобно на числата на Стирлинг от първи род има числа на Лах със и без знак. Числата на Лах без знак обикновено се бележат:
\[\lahfinal{n}{k}, \quad L(n,k), \quad \mathcal{L}_{n,k}.\]

Това са броят начини $n$-елементно множество да се разбие на $k$ на брой непразни линейни наредби. Това е все едно броят начини от азбука с $n$ букви да образуваме $k$ думи без да повтаряме букви, общо използващи целия набор от букви. \\

За начални условия взимаме:
\begin{equation*}
\begin{aligned}
	&\lahfinal{0}{0}=1 \quad \text{(условно приемане)} \\
    &\lahfinal{n}{0}=0, \enspace n>0 \quad \text{(не се получава разбиване)}, \\
    &\lahfinal{n}{n}=1 \quad \text{(всяка буква е дума)}, \\
    &\lahfinal{n}{k}=0, \enspace n<k \quad \text{(не достигат букви)}.
\end{aligned}
\end{equation*}

\begin{comment}
\[
\lahfinal{n}{0}=0 \text{(не зачитаме празната дума)}, \quad \lahfinal{n}{n}=1 \text{(всяка буква е дума)}, \quad \lahfinal{n}{k}=0, \enspace n<k \text{(не достигат букви)}.
\]
\end{comment}

\begin{theorem}
Рекурентната формула, задаваща числата на Лах е:
\[
\lahfinal{n+1}{k}=\left(n+k\right)\lahfinal{n}{k} + \lahfinal{n}{k-1}.
\]
\end{theorem}

\begin{proof}[Док.]
$n+1$-вият елемент или участва в някоя "предишна"\ дума (първото събираемо), или е сам (второто събираемо).
\begin{enumerate}
\item Нека имаме разбиване по линейни наредби, чиито дължини са съответно $i_{1}, ..., i_{k}$. Тогава може да поставим $n+1$-вият елемент или след някоя от тях или най-напред. Но това са съответно $i_{1}+1, ..., i_{k}+1$ начина. Тогава има $\sum_{j=0}^{k} (i_{j}+1)=n+k$.
\item Трябва да получим $k-1$ линейни наредби от останалите $(n+1)-1=n$ елемента.
\end{enumerate}
\end{proof}


\begin{theorem}
Може да се даде явна формула за числата на Лах. По-точно:
\[
\lahfinal{n}{k}=\frac{n!}{k!} \binom{n-1}{k-1}.
\]
\end{theorem}

\begin{proof}[Док.]
Има $n!$ пермутации на $n$ елемента. Думи може да получим, като на всяка пермутация между елементи сложим "разграничител", така че да получим общо $k$ думи, но местата между $n$-те елемента са $n-1$, а за да получим $k$ думи ти трябват $k-1$ разграничителя, откъдето получаваме члена $\binom{n-1}{k-1}$. Така обаче достигнахме брой начини за наредена последователност от $k$ думи, а се интересуваме само от това какви са те. Но за всяко множество от $k$ думи има $k!$ техни пермутации, получени по посочения по-горе начин.
\end{proof}


Нека с $L(n,k)$ бележим коефициента пред $k$-тия намаляващ факториел в "развитието" на $n$-тия растящ факториел, тоест:
\[
x\risingfactorial{n}=\sum_{k=0}^{n} L(n,k) x\fallingfactorial{k}
\]

Ще отбележим, че това е възможно, тъй като отляво и отдясно се получават полиноми от $n$-та степен, а ${x\fallingfactorial{0},..x\fallingfactorial{n}}$ образуват базис за полиномите от $n$-та степен (или приемете това твърдение на доверие, или разгледайте връзката между числата на Стирлинг от втори род и падащите факториели).

\begin{theorem}
В сила е следното тъждество, даващо връзка между числата на Стирлинг от първи и втори род и $L(n,k)$:
\[
L(n,k)=\sum_{j=0}^{n} \stirlingI{n}{j} \stirlingII{j}{k}.
\]
\end{theorem}

\begin{proof}[Док.]
Щом $s\left(n,k\right)$ е коефициента пред $x^{k}$ в $x\fallingfactorial{n}$, то $\stirlingI{n}{k}$ е коефициента пред $x^{k}$ в $x\risingfactorial{k}$ (напр. разглеждане на $(-x)\fallingfactorial{n}$).
Също така $x^n=\sum_{k=0}^{n} \stirlingII{n}{k} x\fallingfactorial{k}$. Използвайки и двете се получава:
\begin{equation*}
\begin{aligned}
    x\risingfactorial{n} &=
\sum_{k=0}^{n} \stirlingI{n}{k} x^{k} =
\sum_{k=0}^{n} \stirlingI{n}{k} \sum_{i=0}^{k} \stirlingII{k}{i} x\fallingfactorial{i} \\
&=
\sum_{k=0}^{n} \left(\sum_{j=k}^{n} \stirlingI{n}{j} \stirlingII{j}{k} \right) x\fallingfactorial{k} =
\sum_{k=0}^{n} \left(\sum_{j=0}^{n} \stirlingI{n}{j} \stirlingII{j}{k} \right) x\fallingfactorial{k}
\end{aligned}
\end{equation*}

\begin{comment}
\[
x\risingfactorial{n} =
\sum_{k=0}^{n} \stirlingI{n}{k} x^{k} =
\sum_{k=0}^{n} \stirlingI{n}{k} \sum_{i=0}^{k} \stirlingII{k}{i} x\fallingfactorial{i} =
\sum_{k=0}^{n} \left(\sum_{j=k}^{n} \stirlingI{n}{j} \stirlingII{j}{k} \right) x\fallingfactorial{k} =
\sum_{k=0}^{n} \left(\sum_{j=0}^{n} \stirlingI{n}{j} \stirlingII{j}{k} \right) x\fallingfactorial{k}
\]
\end{comment}
За последното равенство използваме, че $\stirlingII{n}{k}=0$ при $n<k$.
\end{proof}

Остава ни последното твърдение, а то е:

\begin{theorem}
\[
\lahfinal{n}{k}=L(n,k).
\]
\end{theorem}

\begin{proof}[Док.]
За да докажем, че двете "таблици" \ от числа са равносилно е достатъчно да проверим еднаквостта на началните условия и рекурентните уравнения. \\
Имаме, че:
\begin{equation*}
\begin{aligned}
	&L(0,0)=1 \quad \text{($x\risingfactorial{0}=1=x\fallingfactorial{0}, \enspace 1=1.x^{0}$)}, \\
    &L(n,0)=0, \enspace n>0 \quad \text{(полиномът $x\risingfactorial{n}$ е без свободен член, а $x\fallingfactorial{0}=1.x^{0}$)}, \\
    &L(n,n)=1 \quad \text{(от сумата остава $\stirlingI{n}{n}\stirlingII{n}{n}=1.1=1$)}, \\
    &L(n,k)=0, \enspace n<k  \quad \text{(полиномът $x\fallingfactorial{k}$ е с ненулев член пред $x^{k}$)}.
\end{aligned}
\end{equation*}

Така началните условия съвпадат. За проверка на рекурентната зависимост използваме равенството от по-рано:

\begin{equation*}
\begin{aligned}
    &L(n+1,k) \\ &= \sum_{j=0}^{n+1} \stirlingI{n+1}{j} \stirlingII{j}{k} \quad \text{(прилагаме Тв. 3)} \\
    &= \sum_{j=0}^{n+1} \left(n \stirlingI{n}{j} + \stirlingI{n}{j-1} \right) \stirlingII{j}{k} \quad \text{(от рекурентната зависимост на $\stirlingI{n}{k}$)} \\
    &= n \sum_{j=0}^{n+1} \stirlingI{n}{j} \stirlingII{j}{k} + \sum_{j=0}^{n+1} \stirlingI{n}{j-1} \stirlingII{j}{k}  \\
    &= n \stirlingI{n}{n+1} \stirlingII{n+1}{k} + n \sum_{j=0}^{n} \stirlingI{n}{j} \stirlingII{j}{k} + \stirlingI{n}{-1} \stirlingII{0}{k} + \sum_{j=1}^{n+1} \stirlingI{n}{j-1} \stirlingII{j}{k} \\
	&= n \sum_{j=0}^{n} \stirlingI{n}{j} \stirlingII{j}{k} + \sum_{j=0}^{n} \stirlingI{n}{j} \stirlingII{j+1}{k} \quad \text{(от условията над числата на Стирлинг)} \\
    &= n \sum_{j=0}^{n} \stirlingI{n}{j} \stirlingII{j}{k} + \sum_{j=0}^{n} \stirlingI{n}{j} \left( k \stirlingII{j}{k} + \stirlingII{j}{k-1} \right) \quad \text{(от рекурентната зависимост на $\stirlingII{n}{k}$)} \\
    &= (n+k) \sum_{j=0}^{n} \stirlingI{n}{j} \stirlingII{j}{k} + \sum_{j=0}^{n} \stirlingI{n}{j}  \stirlingII{j}{k-1} \\
    &= (n+k) L(n,k) + L(n,k-1)
\end{aligned}
\end{equation*}
 Сега отчитаме Тв. 1 и явно рекурентата зависимост е същата. И тъй двете "таблици" \ съвпадат.
\end{proof}

Ето и началната част от таблицата на числата на Лах без знак:


\[
\left|
\begin{array}{c|c|c|c|c|c|c|c|c|c|c}
\hline
$\backslashbox{n}{k}$ &
0 & 1 & 2 & 3 & 4 & 5 & 6 & 7 & 8 & 9 \\
\hline
 0 & 1 & 0 & 0 & 0 & 0 & 0 & 0 & 0 & 0 & 0 \\ \hline
 1 & 0 & 1 & 0 & 0 & 0 & 0 & 0 & 0 & 0 & 0 \\ \hline
 2 & 0 & 2 & 1 & 0 & 0 & 0 & 0 & 0 & 0 & 0 \\ \hline
 3 & 0 & 6 & 6 & 1 & 0 & 0 & 0 & 0 & 0 & 0 \\ \hline
 4 & 0 & 24 & 36 & 12 & 1 & 0 & 0 & 0 & 0 & 0 \\ \hline
 5 & 0 & 120 & 240 & 120 & 20 & 1 & 0 & 0 & 0 & 0 \\ \hline
 6 & 0 & 720 & 1800 & 1200 & 300 & 30 & 1 & 0 & 0 & 0 \\ \hline
 7 & 0 & 5040 & 15120 & 12600 & 4200 & 630 & 42 & 1 & 0 & 0 \\ \hline
 8 & 0 & 40320 & 141120 & 141120 & 58800 & 11760 & 1176 & 56 & 1 & 0 \\ \hline
 9 & 0 & 362880 & 1451520 & 1693440 & 846720 & 211680 & 28224 & 2016 & 72 & 1 \\ \hline
\end{array}
\right|
\]

Някои интересни наблюдения:
\begin{comment}
\[
\lahfinal{n}{1} = n! \text{очевидно}, \quad \lahfinal{n}{n-1} = \sum_{i=0}^{n} 2i \text{проста индукция с рекурентата зависимост}, \quad  \lahfinal{n}{k} \left| \lahfinal{n+1}{k+1} \right. ,  \quad \lahfinal{n}{\lfloor \sqrt{n} \rfloor} = \max_{k} \lahfinal{n}{k}
\]
\end{comment}

\begin{equation*}
\begin{aligned}
&\lahfinal{n}{1} = n! \quad \text{(очевидно)}, \\
&\lahfinal{n}{n-1} = \sum_{i=0}^{n-1} 2i = n(n-1) \quad \text{(проста индукция с рекурентата зависимост)}, \\
&\lahfinal{n}{\lfloor \sqrt{n} \rfloor} = \max_{k} \lahfinal{n}{k} \quad \text{(предположение, проверено до n=3000)}, \\
&\lahfinal{n^2-1}{\lfloor \sqrt{n^2-1} \rfloor}=\lahfinal{n^2-1}{\lfloor \sqrt{n^2-1} \rfloor + 1}, \enspace n>1 \quad \text{(доказано долу)}
\end{aligned}
\end{equation*}

\begin{theorem}
\[
\lahfinal{n^2-1}{\lfloor \sqrt{n^2-1} \rfloor}=\lahfinal{n^2-1}{\lfloor \sqrt{n^2-1} \rfloor + 1}, \enspace n>1
\]

\end{theorem}
\begin{proof}[Док.]
Знаем, че $n^2$ е точен квадрат, чийто корен е цялото число $n$. Но тогава $n^2-1$ ще бъде с корен между тeзи на $(n-1)^2$=$n^2-2n+1$ и $n^2$ (понеже $2n-1>1$), но тогава цялата му част ще е $n-1$. Получаваме:

\begin{equation*}
\begin{aligned}
&\lahfinal{n^2-1}{\lfloor \sqrt{n^2-1} \rfloor} = \lahfinal{n^2-1}{n-1} = \frac{(n^2-1)!(n^2-2)!}{(n-1)!(n-2)!(n^2-n)!} \quad \text{(отляво)} \\
&\lahfinal{n^2-1}{\lfloor \sqrt{n^2-1} \rfloor + 1} = \lahfinal{n^2-1}{n} = \frac{(n^2-1)!(n^2-2)!}{n!(n-1)!(n^2-n-1)!} \quad \text{(отдясно)}
\end{aligned}
\end{equation*}

Тяхното отношение е:
\[
\frac{\frac{(n^2-1)!(n^2-2)!}{(n-1)!(n-2)!(n^2-n)!}}{\frac{(n^2-1)!(n^2-2)!}{n!(n-1)!(n^2-n-1)!}} = \frac{n!(n-1)!(n^2-n-1)!}{(n-1)!(n-2)!(n^2-n)!} = \frac{n(n-1)}{n^2-n}=1 \quad \text{(и тъй те са равни)}
\]
\end{proof}

Числата на Лах със знак нямат стандартно означение. Дефинират се като $(-1)^n \lahfinal{n}{k}$. Оказва се, че те се явяват като коефициенти пред $x^{-n-k}$ при $n$-тата производна на $e^{\frac{1}{x}}$, разделена на $e^{\frac{1}{x}}$.[\href{http://math.pugetsound.edu/~mspivey/Exp.pdf}{Вижте тук}]

\end{large}
\end{document}
