\setcounter{equation}{0}
\section{Гравитационни вълни}
Хидродинамиката разглежда 3 вида вълни - повърхнинни, вътрешни и свивателни.
И в трите случая вълните са предвижващи се флуидни трептения.
При първия тип връщащи сили се оказват повърхностното напрежение и гравитацията,
при вторите - само гравитацията, а при третите - от свиваемостта на флуида.
Вълните на границата на въздух-вода са добър пример за повърхнинни вълни и се наричат водни вълни.

\subsection{Основни понятия}
Нека разглеждаме координатна система с абсциса $x$ и ордината абсциса $z$.
Синусуидална бягаща вълна се представя с уравнението
\begin{equation}
	z(x,t) = A \cos \left[\frac{2 \pi}{\lambda}(x - c t)\right]
\end{equation}
Тук $A$ е амплитудата на вълната - най-голямото отдалечаване от положението.
Ако фиксираме времето $t$, менейки $x$, то имаме следваща амплитуда при:
\begin{equation}
	\frac{2 \pi}{\lambda}((x + \lambda) - c t) = \frac{2 \pi}{\lambda}(x - c t) + 2 \pi
\end{equation}
$\lambda$ е дължината на вълната - разстоянието между две съседни амплитуди.
$k = \frac{2 \pi}{\lambda}$ е пространствената честота.
$c$ се нарича фазова скорост, а $2 \pi (x - c t)$ - фаза.
Ако фиксираме $x$, то менейки $t$ получаваме прериодично движение нагоре-надолу.
Ако в $x$ има пик, то следващият ще настъпи, когато към аргумента на $\cos$ добавим $2 \pi$.
Но 
\begin{equation}
	\frac{2 \pi}{\lambda}(x - c t) + 2 \pi = \frac{2 \pi}{\lambda}(x - c t + \lambda) = \frac{2 \pi}{\lambda}(x - c (t + \frac{\lambda}{c}))
\end{equation}
Това води до $T = \frac{\lambda}{c}$ - период между двата пика за фиксираната позиция.
$\nu = \frac{1}{T}$ се нарича циклична честота, а $\omega = 2 \pi \nu = k c$ - радиална честота.
Така може да сведем уравнението на вълната до:
\begin{equation}
	z(x,t) = A \cos (k x - \omega t)
\end{equation}
Пиковете се предвижват по следния начин с времето:
\begin{equation}
	k \xi - \omega t = 2 n \pi, \quad \xi = \frac{\omega}{k} t + \frac{2 n \pi}{k} 
\end{equation}
Така фазовата скорост $c$ се оказва точно скоростта с която се предвижва вълната в пространството за даден период от време. 
Ако $c$ зависи от вълновото число $k$ (или еквивалентно от дължината на вълната $\lambda$), то наричаме повърхностните вълни дисперсни.


\subsection{Групова скорост}
Нека съберем синусоидални вълни с еднакви амплитуди и близки възлови числа. Тогава:
\begin{align}
	z(x, t) = A \cos (k_1 x - \omega_1 t) + A \cos (\delta k_2 x - \omega_2 t) = 2 A \cos (\frac{\Delta k}{2} x - \frac{\Delta \omega}{2} t) \cos (k x - \omega t)
\end{align}
Въвели сме $\Delta k = k_2 - k_1$, $\Delta \omega = \omega_2 - \omega_1$, $k = \frac {k_1 + k_2}{2}$, $\omega = \frac{\omega_1 + \omega_2}{2}$
Да разгледаме вълните, които в пространствен участък имат най-голяма амплитуда, т.е.:
\begin{align}
	&\cos (\frac{\Delta k}{2} \xi - \frac{\Delta \omega}{2} t) \approx 1 \\
	&\frac{\Delta k}{2} \xi - \frac{\Delta \omega}{2} t = n \pi
\end{align}
Скоростта с която се движат точките $\xi$ на максимална амплитуда е точно $\frac{\Delta \omega}{\Delta k}$.
Това води до понятието групова скорост - $c_g = \dv{\omega}{k}$.
Обвивката на вълните с максимална амплитуда в даден момент се движи като вълна със скорост $c_g$.
Aко вълните не са дисперсивни, то имаме, че:
\begin{equation}
	c_g = \dv{\omega}{k} = \dv{c k}{k} = c
\end{equation}
Тоест съвпада със фазовата скорост на отделните вълни. Пример за това са звуковите вълни.
Ако пък разглеждаме вълни около повърхността, то може да се покаже, че $\omega^2 = g k$, тогава:
\begin{equation}
	c_g = \dv{\omega}{k} = \dv{\sqrt{g k}}{k} = \frac{\sqrt{g}}{2 \sqrt{k}} = \frac{c}{2}
\end{equation}
Груповата скорост се оказва два пъти по-малка от тази на отделните вълни.
Изразявайки $\omega, k$ по дефинициите им, то достигаме до формулата на Рейли:
\begin{equation}
	c_g = \dv{\omega}{k} = \dv{\frac{2 \pi c}{\lambda}}{\frac{2 \pi}{\lambda}} = \frac{\lambda \dd c - c \dd \lambda}{- \dd \lambda} = c - \lambda \dv{c}{\lambda}
\end{equation}