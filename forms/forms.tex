\documentclass[12pt]{article}
\usepackage[a4paper, includeheadfoot, margin = 1.5cm]{geometry}
\usepackage[unicode=true, colorlinks=true, linkcolor=black, urlcolor=black]{hyperref}
\usepackage[T2A]{fontenc}
\usepackage[utf8]{inputenc}
\usepackage[bulgarian]{babel}
\usepackage{csquotes}
\usepackage{indentfirst}
\usepackage{amsmath}
\usepackage{amssymb}
\usepackage{mathtools}
\usepackage{comment}
\usepackage{mathptmx}
\usepackage{tikz}
\usepackage{bm}
\usepackage{enumitem}
\usepackage{amsthm}
\usepackage{physics}
\usepackage{derivative}
\usepackage{amsmath,amssymb}

%\fontsize{16pt}{20pt}\selectfont
\renewcommand{\sfdefault}{cmss}
\renewcommand{\rmdefault}{cmr}
\renewcommand{\ttdefault}{cmt}

\renewcommand{\thesection}{\Roman{section}} 
%\renewcommand{\thesubsection}{\thesection.\Roman{subsection}}

\newtheorem{definition}{Дефиниция}[section]
\newtheorem{problem}{Задача}
\newtheorem{theorem}{Теорема}
\newtheorem*{theorem*}{Теорема}
\newtheorem{lemma}{Лема}
\newtheorem*{solution*}{Решение}

%\newcommand\abs[1]{\left \lvert #1  \right \rvert}
\newcommand\numberthis{\addtocounter{equation}{1}\tag{\theequation}}
\newcommand\func[3]{#1:#2 \to #3}
\newcommand\stfunc[1]{\func{#1}{\mathbb{R}^n}{\mathbb{R}}}
\newcommand\nstfunc[1]{\func{#1}{V}{F}}
\newcommand\myxi[0]{\boldsymbol{\xi}}
\newcommand\myeta[0]{\boldsymbol{\eta}}

\DeclarePairedDelimiterX{\dotpr}[2]{\langle}{\rangle}{#1, #2}



%\newcommand{\choose}[3]{\genfrac{\left(}{\right)}{0pt}{#1}{#2}{#3}}

\title{Външни и диференциални форми в $\mathbb{R}^n$}

\author{Калоян Стоилов}

\begin{document}

\maketitle
\begin{large}
\tableofcontents{}


\section{Външни форми}
\begin{definition}[$k$-форма] Нека $F$ е поле, с $char F \neq 2$, а $V$ е векторно пространство с размерност $n$(не задължително над $F$). Външна $k$-форма (или просто $k$-форма) $\omega^k$ е изображение $\func{\omega^k}{V}{F}$, със следните свойства:
\begin{enumerate}
\item Полилинейност: \\ 
$\omega^k(\myxi_1,\cdots,\lambda\myxi_i+\mu\myeta_i,\cdots,\myxi_k)=\lambda\omega^k(\myxi_1,\cdots,\myxi_i,\cdots,\myxi_k)+\mu\omega^k(\myxi_1,\cdots,\myeta_i,\cdots,\myxi_k)$ \\
$\lambda,\mu \in F, i=\overline{1,k}$
\item Антикомутативност: \\
$\omega^k(\myxi_{\pi(1)},\cdots,\myxi_{\pi(k)})=sgn(\pi)\omega^k(\myxi_1,\cdots,\myxi_k)$,където $\pi$ е някаква пермутация.
\end{enumerate}
\end{definition}

$k$-формите също са наричани (външни) форми от степен $k$.

%Ще разглеждаме само форми с $F=\mathbb{R}, \enspace V=\mathbb{R}^n$. 

\subsection{1-форми}

Това са линейните функции от вида $\nstfunc{\omega}$.
Тези фунцкии образуват векторно пространство, ако въведем следните операции между тях:

\begin{align*}
&(\omega_1+\omega_2)(\myxi)=\omega_1(\myxi)+\omega_2(\myxi) \\
&(\lambda\omega_1)(\myxi)=\lambda\omega_1(\myxi) \\
\end{align*}

Очевидно $\omega_1+\omega_2$ и $\lambda\omega_1$ също са линейни и са със същите дефиниционна област и кодомейн. В общият случай за $k$-форми ще докажем, че се получават форми по по-горната дефиниция. Останалите свойства за векторни пространства директно следват от тези на кодомейна, който е поле.\\

Това векторно пространство се нарича спрегнато (дуално) пространство на $V$ и се бележи с $V^{*}$. $1$-формите понякога се наричат ковектори. 

Нека с $x_i(\myxi)$ отбележим проекцията на $\myxi$ върху $i$-тия базисен вектор. Така виждаме, че  самите $x_i$ са линейно независими 1-форми (а ако имаме евклидова структура даже $x_i (\mathbf{x}_j) = \delta_{ij}$). Те образуват базис за $V^{*}$, тъй като всяка линейна фунцкия $\nstfunc{\omega}$ се представя в следния вид: 
\begin{equation*}
\omega(\myxi) = \sum_{i=1}^{n} a_{i} \xi_{i} = \sum_{i=1}^{n} a_{i} x_{i} (\myxi), \quad \myxi=(\xi_1,\cdots,\xi_n)^{T}
\end{equation*}
Този базис се нарича спрегнат (дуален) и често се бележи с $\mathbf{x}^i$. Оказва се, че ако $A=(\mathbf{x}_1,\cdots,\mathbf{x}_n)$, а $B=(\mathbf{x}^1,\cdots,\mathbf{x}^n)$, то $А^T B = E_n$.

Така виждаме, че 1-формите над $\mathbb{R}^n$ могат да се представят чрез скаларно произведение: $\omega_{\mathbf{a}}(\myxi)=\dotpr{\mathbf{a}}{\myxi}, \quad \mathbf{a} \in \mathbb{R}^n$ 


\subsection{2-форми}
Аналогично може да дефинираме операции върху 2-формите:
\begin{align*}
&(\omega_1+\omega_2)(\myxi_1,\myxi_2)=\omega_1(\myxi_1,\myxi_2)+\omega_2(\myxi_1,\myxi_2) \\
&(\lambda\omega_1)(\myxi_1,\myxi_2)=\lambda\omega_1(\myxi_1,\myxi_2) \\
\end{align*}

След аналогични разсъждения, отново получаваме, че те образуват векторно пространство.

\subsection{k-форми}
 Ще покажем, че наистина щом $\omega_1, \omega_2$ са $k$-форми, то $(\omega_1+\omega_2)$ и $(\lambda\omega_1)$ също са $k$-форми:
\begin{align*}
&(\omega_1+\omega_2)(\myxi_1,\cdots,\lambda\myxi_i+\mu\myeta_i,\cdots,\myxi_k) \\
=&\omega_1(\myxi_1,\cdots,\lambda\myxi_i+\mu\myeta_i,\cdots,\myxi_k)+\omega_2(\myxi_1,\cdots,\lambda\myxi_i+\mu\myeta_i,\cdots,\myxi_k) \\
=&\lambda\omega_1(\myxi_1,\cdots,\myxi_i,\cdots,\myxi_k)+\mu\omega_1(\myxi_1,\cdots,\myeta_i,\cdots,\myxi_k)\\
&+\lambda\omega_2(\myxi_1,\cdots,\myxi_i,\cdots,\myxi_k)+\mu\omega_2(\myxi_1,\cdots,\myeta_i,\cdots,\myxi_k) \\
=&\lambda(\omega_1(\myxi_1,\cdots,\myxi_i,\cdots,\myxi_k)\omega_2(\myxi_1,\cdots,\myxi_i,\cdots,\myxi_k))\\
&+\mu(\omega_1(\myxi_1,\cdots,\myeta_i,\cdots,\myxi_k)+\omega_2(\myxi_1,\cdots,\myeta_i,\cdots,\myxi_k)) \\
=&\lambda (\omega_1+\omega_2)(\myxi_1,\cdots,\myxi_i,\cdots,\myxi_k) + \mu (\omega_1+\omega_2)(\myxi_1,\cdots,\myeta_i,\cdots,\myxi_k) \\
&(\omega_1+\omega_2)(\myxi_{\pi(1)},\cdots,\myxi_{\pi(k)})=\omega_1(\myxi_{\pi(1)},\cdots,\myxi_{\pi(k)})+\omega_2(\myxi_{\pi(1)},\cdots,\myxi_{\pi(k)}) \\
=&sgn(\pi)\omega_1(\myxi_1,\cdots,\myxi_i,\cdots,\myxi_k)+sgn(\pi)\omega_2(\myxi_1,\cdots,\myxi_i,\cdots,\myxi_k) \\
=&sgn(\pi)(\omega_1(\myxi_1,\cdots,\myxi_i,\cdots,\myxi_k)+\omega_2(\myxi_1,\cdots,\myxi_i,\cdots,\myxi_k))\\
=&sgn(\pi)(\omega_1+\omega_2)(\myxi_1,\cdots,\myxi_i,\cdots,\myxi_k) \\
&(\lambda\omega_1)(\myxi_1,\cdots,\theta\myxi_i+\mu\myeta_i,\cdots,\myxi_k)=\lambda\omega_1(\myxi_1,\cdots,\theta\myxi_i+\mu\myeta_i,\cdots,\myxi_k) \\
=&\lambda(\theta\omega_1(\myxi_1,\cdots,\myxi_i,\cdots,\myxi_k)+\mu\omega_1(\myxi_1,\cdots,\myeta_i,\cdots,\myxi_k)) \\
=&\theta(\lambda\omega_1(\myxi_1,\cdots,\myxi_i,\cdots,\myxi_k))+\mu(\lambda\omega_1(\myxi_1,\cdots,\myeta_i,\cdots,\myxi_k)) \\
=&\theta(\lambda\omega_1)(\myxi_1,\cdots,\myxi_i,\cdots,\myxi_k)+\mu(\lambda\omega_1)(\myxi_1,\cdots,\myeta_i,\cdots,\myxi_k) \\
&(\lambda\omega_1)(\myxi_{\pi(1)},\cdots,\myxi_{\pi(k)})=\lambda\omega_1(\myxi_{\pi(1)},\cdots,\myxi_{\pi(k)}) \\
=&sgn(\pi)(\lambda\omega_1 (\myxi_1,\cdots,\myxi_i,\cdots,\myxi_k))=sgn(\pi)(\lambda\omega_1)(\myxi_1,\cdots,\myxi_i,\cdots,\myxi_k) \\
\end{align*}

Сега вече сме сигурни, че $k$-формите образуват векторно пространство. Тъй като
\begin{align*}
&\omega^k(\myxi_1,\cdots,\myxi_i,\cdots,\myxi_i,\cdots\myxi_k)=-\omega^k(\myxi_1,\cdots,\myxi_i,\cdots,\myxi_i,\cdots\myxi_k), \quad i \neq j \\
&2\omega^k(\myxi_1,\cdots,\myxi_i,\cdots,\myxi_i,\cdots\myxi_k)=0 \\
\end{align*}

\begin{comment}
тo в зависимост от $F$ има два случая:
\begin{enumerate}
\item $F$ е крайно, то  $\omega^k(\myxi_1,\cdots,\myxi_i,\cdots,\myxi_i,\cdots\myxi_k) \in \left\{a\in F \lvert a+a=0 \right\}$
\item $F$ е безкрайно, то  $\omega^k(\myxi_1,\cdots,\myxi_i,\cdots,\myxi_i,\cdots\myxi_k) \in \left\{a\in F \lvert a+a=0 \right\}$
\end{enumerate}
\end{comment}
Тъй като  при полета няма делители на нулата в мултипликативната група, то:
\begin{align*}
&(1+1)\omega^k(\myxi_1,\cdots,\myxi_i,\cdots,\myxi_i,\cdots\myxi_k)=0 \\
&\Longleftrightarrow (1+1=0) \lor (\omega^k(\myxi_1,\cdots,\myxi_i,\cdots,\myxi_i,\cdots\myxi_k)=0)
\end{align*}
Но по дефиниця полето $F$ е такова, че $char F \neq 2$, т.е. $1+1 \neq 0$. Но тогава:
\[\omega^k(\myxi_1,\cdots,\myxi_i,\cdots,\myxi_i,\cdots\myxi_k)=0
\]

\subsection{Външно умножение}
Въвеждаме операцията външно множение $\wedge$ над $1$-формите по следният начин:
\[
(\omega_1 \wedge \cdots \wedge \omega_k)(\myxi_1,\cdots,\myxi_k)=
\begin{vmatrix}
&\omega_1(\myxi_1) &\cdots &\omega_1(\myxi_k) \\
&\vdots &\ddots &\vdots \\
&\omega_k(\myxi_1) &\cdots &\omega_k(\myxi_k) 
\end{vmatrix} 
\]
От $\det A = \det A^T$, то виждаме и друга еквивалентна дефиниция. Наистина получаваме $k$-форма:
\begin{align*}
&(\omega_1 \wedge \cdots \wedge \omega_k)(\myxi_1,\cdots,\lambda\myxi_i+\mu\myeta_i,\cdots,\myxi_k)=
\begin{vmatrix}
&\omega_1(\myxi_1) &\cdots &\omega_1(\lambda\myxi_i+\mu\myeta_i) &\cdots &\omega_1(\myxi_k) \\
&\vdots &\ &\vdots &\ &\vdots \\
&\omega_k(\myxi_1) &\cdots &\omega_k(\lambda\myxi_i+\mu\myeta_i) &\cdots &\omega_k(\myxi_k) 
\end{vmatrix} \\
&=\begin{vmatrix}
&\omega_1(\myxi_1) &\cdots &\lambda\omega_1(\myxi_i)+\mu\omega_1(\myeta_i) &\cdots &\omega_1(\myxi_k) \\
&\vdots &\ &\vdots &\ &\vdots \\
&\omega_k(\myxi_1) &\cdots &\lambda\omega_k(\myxi_i)+\mu\omega_k(\myeta_i) &\cdots &\omega_k(\myxi_k) 
\end{vmatrix} \text{(линейност на $1$-форми)} \\
&=\lambda\begin{vmatrix}
&\omega_1(\myxi_1) &\cdots &\omega_1(\myxi_i) &\cdots &\omega_1(\myxi_k) \\
&\vdots &\ &\vdots &\ &\vdots \\
&\omega_k(\myxi_1) &\cdots &\omega_k(\myxi_i) &\cdots &\omega_k(\myxi_k) 
\end{vmatrix}+\mu
\begin{vmatrix}
&\omega_1(\myxi_1) &\cdots &\omega_1(\myeta_i) &\cdots &\omega_1(\myxi_k) \\
&\vdots &\ &\vdots &\ &\vdots \\
&\omega_k(\myxi_1) &\cdots &\omega_k(\myeta_i) &\cdots &\omega_k(\myxi_k) 
\end{vmatrix} \underset{\text{на дет.)}}{\text{(свойства}} \\
&=\lambda(\omega_1 \wedge \cdots \wedge \omega_k)(\myxi_1,\cdots,\myxi_i,\cdots,\myxi_k)+\mu(\omega_1 \wedge \cdots \wedge \omega_k)(\myxi_1,\cdots,\myeta\myeta_i,\cdots,\myxi_k) \\
&(\omega_1 \wedge \cdots \wedge \omega_k)(\myxi_{\pi(1)},\cdots,\myxi_{\pi(k)})=sgn(\pi)(\omega_1 \wedge \cdots \wedge \omega_k)(\myxi_1,\cdots,\myxi_k) \text{(директно от св. на дет.)}
\end{align*}

$k$-формите, които се представят по този начин, т.е. $\omega^k=\omega^1_1 \wedge \cdots \wedge \omega^1_k$ се наричат разложими.

Очевидно $(\omega_1 \wedge \cdots \wedge \omega_i \wedge \cdots \wedge \omega_i \wedge \cdots \wedge \omega_k)(\myxi_1,\cdots,\myxi_k)=0, \quad i \neq j$. 
Нека $\mathbf{x}_1,\cdots ,\mathbf{x}_n$ е базис в $\mathbb{R}^n$. Той определя $n$ на брой $1$-форми. От тях може да изберем $n \choose k$ различни и линейно независими разложими $k$-форми, като приемем, че избираме от базисните $1$-форми с растящ индекс (иначе или бихме получили тъждествена на $0$ форма, или че формата е $-1.$някоя от тези).
От друга страна, поради линейността на формите те се определят от действието си над всяка последователност от базисните вектори. Но тъй като са антикомутативни, то еднозначно се определят от действието над $(\mathbf{x}_{i_1}, \cdots, \mathbf{x}_{i_k})$, където $i_1<\cdots<i_k$. Но за всяка такава последователност има форма $x_{i_1} \wedge \cdots \wedge x_{i_k}$. Веднага от дефиницята следва, че:
\begin{comment} b_{i_1,\cdots,i_k} \quad (\textbf{при евклидова структура $b_{i_1,\cdots,i_k}=1$}) 
\end{comment}
\begin{align*}
&x_{i_1} \wedge \cdots \wedge x_{i_k}(\mathbf{x}_{i_1}, \cdots, \mathbf{x}_{i_k}) = 1  \\
&x_{i_1} \wedge \cdots \wedge x_{i_k}(\mathbf{x}_{j_1}, \cdots, \mathbf{x}_{j_k}) = 0, \quad (i_1,\cdots,i_k)\neq(j_1,\cdots,j_k)
\end{align*}
Но така всяка $k$-форма е представима по начина:
\[
\omega^k = \sum_{i_1,\cdots,i_k} a_{i_1,\cdots,i_k} x_{i_1} \wedge \cdots \wedge x_{i_k}
\]

В $R^n$ има единствена $n$-форма в базиса - $\mu_n = x_1 \wedge \cdots \wedge x_n$, която се нарича форма на обема, която отговаря на ориентираният обем, заключен между векторите, върху които е приложена. \\ 

Сега дефинираме операцията $\wedge$ над разложимите форми. Нека $\omega^k=\omega_1 \wedge \cdots \wedge \omega_k$, a $\omega^l=\omega_{k+l} \wedge \cdots \wedge \omega_{k+l}$. Тогава $\omega^k \wedge \omega^l = \omega_1 \wedge \cdots \wedge \omega_k \wedge \omega_{k+1} \wedge \cdots \wedge \omega_{k+l}$. В сила е, че:
\begin{align*}
&(\omega^k \wedge \omega^l) \wedge \omega^m = (\omega_1 \wedge \cdots \wedge \omega_k \wedge \omega_{k+1} \wedge \cdots \wedge \omega_{k+l}) \wedge \omega^m \\ 
&=\omega_1 \wedge \cdots \wedge \omega_k \wedge \omega_{k+1} \wedge \cdots \wedge \omega_{k+l} \wedge \omega_{k+l+1} \wedge \cdots \wedge \omega_{k+l+m} \\
&= \omega^k \wedge (\omega_{k+1} \wedge \cdots \wedge \omega_{k+l} \wedge \omega_{k+l+1} \wedge \cdots \wedge \omega_{k+l+m}) \\
&=\omega^k \wedge (\omega^l \wedge \omega^m) \\[20pt]
&\omega^k \wedge \omega^l=\omega_1 \wedge \cdots \wedge \omega_k \wedge \omega_{k+1} \wedge \cdots \wedge \omega_{k+l} \\
&=(-1)^{kl}\omega_{k+1} \wedge \cdots \wedge \omega_{k+l} \wedge \omega_1 \wedge \cdots \wedge \omega_k = (-1)^{kl} \omega^l \wedge \omega^k \\
&\text{тъй като има инверсии между всяко от $1,\cdots,k$ и всяко от $k+1,\cdots,k+l$}
\end{align*}

Може да се въведе и операцията $\wedge$ върху произволни форми по следния начин: 
\[
(\omega^k \wedge \omega^l)(\myxi_1,\cdots,\myxi_{k+l})=\mathop{\sum_{i_1<\cdots<i_k}}_{j_1<\cdots<j_l}sgn(i_1,\cdots,i_k,j_1, \cdots,j_l) \omega^k(\myxi_{i_1},\cdots,\myxi_{i_k})\omega^l(\myxi_{j_1},\cdots,\myxi_{j_l})
\]

Резултатът е $k+l$-форма:\\
Нека $\myxi_q=\lambda\myeta_q^1+\mu\myeta_q^2$. В зависимост от пермутацията, то $\myxi_q$ се намира или сред аргументите на $\omega^k$ или сред тези на $\omega^l$. Ще покажем, ако е сред тези на $\omega^k$(другият случай е аналогичен). Тогава събираемите са от вида:
\begin{align*}
&\omega^k(\myxi_{i_1},\cdots,\lambda\myeta_q^1+\mu\myeta_q^2 ,\cdots,\myxi_{i_k})\omega^l(\myxi_{j_1},\cdots,\myxi_{j_l}) \\
&=(\lambda\omega^k(\myxi_{i_1},\cdots,\myeta_q^1,\cdots,\myxi_{i_k}) + \mu\omega^k(\myxi_{i_1},\cdots,\myeta_q^2,\cdots,\myxi_{i_k}))\omega^l(\myxi_{j_1},\cdots,\myxi_{j_l}) \\
&=\underbrace{\lambda\omega^k(\myxi_{i_1},\cdots,\myeta_q^1,\cdots,\myxi_{i_k})\omega^l(\myxi_{j_1},\cdots,\myxi_{j_l})
}_{\text{присъства в сумата на $\lambda(\omega^k \wedge \omega^l)$}}+
\underbrace{\mu\omega^k(\myxi_{i_1},\cdots,\myeta_q^2,\cdots,\myxi_{i_k})\omega^l(\myxi_{j_1},\cdots,\myxi_{j_l})
}_{\text{присъства в сумата на $\mu(\omega^k \wedge \omega^l)$}}
\end{align*}
За по-кратко сме пропуснали общия множител от четността на пермутацията. Тъй като аналогичното е в сила и при $\omega^l$, а във всеки от членовете на сумите $q$ присъства в една от двете форми, то след разпределение на общият множител по сумите, прегрупиране и изнасяне на $\lambda$ и $\mu$ пред крайните суми, получаваме:
\begin{align*}
&(\omega^k \wedge \omega^l)(\myxi_1,\cdots,\lambda\myeta_q^1+\mu\myeta_q^2,\cdots,\myxi_{k+l})=\\
&\lambda(\omega^k \wedge \omega^l)(\myxi_1,\cdots,\myeta_q^1,\cdots,\myxi_{k+l})+ \mu(\omega^k \wedge \omega^l)(\myxi_1,\cdots,\myeta_q^2,\cdots,\myxi_{k+l})
\end{align*}
Остава да приверим антикомутативността:
\begin{align*}
&(\omega^k \wedge \omega^l)(\myxi_{\pi(1)},\cdots,\myxi_{\pi(k+l)})= sgn(\pi)(\omega^k \wedge \omega^l)(\myxi_1,\cdots,\myxi_{k+l}) \\
\end{align*}
Тя следва от факта, че в събираемите ще композираме тази фиксирана пермутация $\pi$ и съответната за събираемото, да кажем $\sigma$. Използваме, че $sgn(\sigma \circ \pi)=sgn(\sigma)sgn(\pi)$.
\begin{enumerate}[label=\alph*)]
\item $sgn(\pi)=1$: Тогава $sgn(\sigma \circ \pi)=sgn(\sigma)$ и получаваме същите събираеми.
\item $sgn(\pi)=-1$: Тогава $sgn(\sigma \circ \pi)=-sgn(\sigma)$ и получаваме събираемите, но с обратен знак.
\end{enumerate}
И в двата случая може да изкараме общия знак пред сумата, но той съвпада с $sgn(\pi)$.

Свойства на $\wedge$:
\begin{enumerate}
\item Дистрибутивност: $(\lambda\omega^k_1+\mu\omega^k_2) \wedge \omega^l = \lambda\omega^k_1\wedge\omega^l + \mu \omega^k_2\wedge\omega^l$ \\
Събираемите в сумата(без знака за пермутация) са от вида:
\begin{align*}
&(\lambda\omega^k_1+\mu\omega^k_2)(\myxi_{i_1},\cdots,\myxi_{i_k})\omega^l(\myxi_{j_1},\cdots,\myxi_{j_l}) \\
&=(\lambda\omega^k_1(\myxi_{i_1},\cdots,\myxi_{i_k})+\mu\omega^k_2(\myxi_{i_1},\cdots,\myxi_{i_k}))\omega^l(\myxi_{j_1},\cdots,\myxi_{j_l}) \\
&=\underbrace{\lambda\omega^k_1(\myxi_{i_1},\cdots,\myxi_{i_k})\omega^l(\myxi_{j_1},\cdots,\myxi_{j_l})}_{\text{присъства в сумата на $\lambda\omega^k_1\wedge\omega^l$}}+
\underbrace{\mu\omega^k_2(\myxi_{i_1},\cdots,\myxi_{i_k})\omega^l(\myxi_{j_1},\cdots,\myxi_{j_l})}_{\text{присъства в сумата на $\mu\omega^k_2\wedge\omega^l$}}
\end{align*}
След повтаряне на предишните разсъждения получаваме търсеното.
\item Асоциативност $(\omega^k \wedge \omega^l) \wedge \omega^m=\omega^k \wedge (\omega^l \wedge \omega^m)$
\begin{align*}
&(\omega^k \wedge \omega^l) \wedge \omega^m  \\ 
&=\mathop{\sum_{r_1<\cdots<r_{k+l}}}_{p_1<\cdots<p_m} sgn(r_1,\cdots,r_{k+l},p_1, \cdots,p_m)(\omega^k \wedge \omega^l)(\myxi_{r_1},\cdots,\myxi_{r_{k+l}})\omega^m(\myxi_{p_1},\cdots,\myxi_{p_m}) \\
&=\mathop{\sum_{r_1<\cdots<r_{k+l}}}_{p_1<\cdots<p_m} sgn(r_1,\cdots,r_{k+l},p_1, \cdots,p_m) \\ &\left(\mathop{\sum_{i_1<\cdots<i_k}}_{j_1<\cdots<j_l} sgn(i_1,\cdots,i_k,j_1, \cdots,j_l)\omega^k(\myxi_{i_1},\cdots,\myxi_{i_k})\omega^l(\myxi_{j_1},\cdots,\myxi_{j_l}) \right)\omega^m(\myxi_{p_1},\cdots,\myxi_{p_m}) \\
&=\mathop{\mathop{\sum_{i_1<\cdots<i_k}}_{j_1<\cdots<j_l}}_{p_1<\cdots<p_m} sgn(I,J,P) \omega^k(\myxi_{i_1},\cdots,\myxi_{i_k})\omega^l(\myxi_{j_1},\cdots,\myxi_{j_l})\omega^m(\myxi_{p_1},\cdots,\myxi_{p_m}) \\
&= \omega^k \wedge (\omega^l \wedge \omega^m), \quad (I,J,P)=(i_1,\cdots,i_k,j_1,\cdots,j_l,p_1,\cdots,p_m)
\end{align*}
\item Антикомутативност $\omega^k \wedge \omega^l=(-1)^{kl}\omega^l \wedge \omega^k$
\begin{align*}
&\omega^k \wedge \omega^l=\mathop{\sum_{i_1<\cdots<i_k}}_{j_1<\cdots<j_l} sgn(i_1,\cdots,i_k,j_1, \cdots,j_l)\omega^k(\myxi_{i_1},\cdots,\myxi_{i_k})\omega^l(\myxi_{j_1},\cdots,\myxi_{j_l})\\
&=\mathop{\sum_{j_1<\cdots<j_l}}_{i_1<\cdots<i_k} (-1)^{kl}sgn(j_1, \cdots,j_l,i_1,\cdots,i_k)\omega^l(\myxi_{j_1},\cdots,\myxi_{j_l})\omega^k(\myxi_{i_1},\cdots,\myxi_{i_k}) \\
&=(-1)^{kl}\omega^l \wedge \omega^k
\end{align*}
\end{enumerate}

Нека $\omega$ е $k$-форма, а с $\omega^m$ отбележим $\underbrace{\omega \wedge \cdots \wedge \omega}_{\text{m пъти}}$. Тогава е в сила, че:
\begin{itemize}
\item Ако $k$ е нечетно, то $\omega^m=0, \enspace m>1$:
\begin{align*}
&m=2: \omega \wedge \omega=(-1)^{k^2}\omega \wedge \omega=-\omega \wedge \omega \implies \omega \wedge \omega=0 \\ 
&m>2: \omega^m=\omega \wedge \omega \wedge \omega^{m-2}=(\omega \wedge \omega)\wedge \omega^{m-2}=0 \wedge \omega^{m-2}=0
\end{align*}
\item Ако $\omega$ е разложима, то $\omega^m=0, \enspace m>1$:
\begin{align*}
&\omega=\omega_1 \wedge \cdots \omega_k, \quad  \text{$\omega_j$ са $1$-форми} \\
m=2: &\omega^2=\omega_1 \wedge \cdots \omega_k \wedge \omega_1 \wedge \cdots \omega_k = (-1)^{(k-1)k^2} (\omega_1 \wedge \omega_1) \wedge \omega_2 \wedge \cdots \omega_k \wedge \omega_2 \wedge \cdots \omega_k \\
&=0 \wedge \omega_2 \wedge \cdots \omega_k \wedge \omega_2 \wedge \cdots \omega_k = 0 \\
m>2: &\omega^m=\omega^2 \wedge \omega^{m-2}=0 \wedge \omega^{m-2}=0
\end{align*}
\end{itemize}

С ${\bigwedge}^k V$ бележим множеството от всички $k$-форми над $V$.

\begin{problem}
Нека над $\mathbb{R}^{2n}=(\mathbf{q},\mathbf{f})$ е дефинирана $2$-формата $\omega=\sum_{i=1}^n p_i \wedge q_i$ Да се намери формата $\omega^m$.
\end{problem}

\begin{solution*}
\begin{align*}
&\omega^1 =  (-1)^0 1! \sum_{i=1}^n p_i \wedge q_i \\
&\omega^{m+1} = \omega \wedge \omega^m = (-1)^{\frac{m(m-1)}{2}} m! (\sum_{i=1}^n p_i \wedge q_i) \wedge (\sum_{i_1<\cdots<i_m}^n p_{i_1} \wedge \cdots  \wedge p_{i_m} \wedge q_{i_1} \wedge \cdots \wedge q_{i_m}) \\
&=(-1)^{\frac{m(m-1)}{2}}m! \sum_{i_1<\cdots<i_m} \sum_{j=1}^n p_j \wedge q_j \wedge p_{i_1} \wedge \cdots  \wedge p_{i_m} \wedge q_{i_1} \wedge \cdots \wedge q_{i_m} \\
&=(-1)^{\frac{m(m-1)}{2}}m! \sum_{i_1<\cdots<i_m} \sum_{j=1}^n (-1)^{m}  p_j  \wedge p_{i_1} \wedge \cdots \wedge p_{i_m} \wedge q_j \wedge q_{i_1} \wedge \cdots \wedge q_{i_m} \\
&=(-1)^{\frac{m(m-1)}{2}}m! \sum_{i_1<\cdots<i_m} \sum_{j \not \in \{i_1,\cdots, i_m\}} (-1)^{m}  p_j  \wedge p_{i_1} \wedge \cdots \wedge p_{i_m} \wedge q_j \wedge q_{i_1} \wedge \cdots \wedge q_{i_m} \\
&=(-1)^{\frac{m(m-1)}{2}+m}m! \mathop{\sum_{i_1<\cdots<i_m}}_{j \not \in \{i_1,\cdots, i_m\}} (sgn(j,i_1,\cdots,i_m))^2  \underbrace{p_{i_1} \wedge \cdots \wedge p_j  \wedge \cdots \wedge p_{i_m} \wedge q_{i_1} \wedge \cdots \wedge q_j  \wedge  \cdots \wedge q_{i_m}}_{i_1<\cdots j < \cdots < i_m} \\
&=(-1)^{\frac{(m+1)m}{2}}m!(m+1)\sum_{i_1<\cdots<i_{m+1}} p_{i_1} \wedge \cdots  \wedge p_{i_{m+1}} \wedge q_{i_1} \wedge \cdots \wedge q_{i_{m+1}} \\
&=(-1)^{\frac{(m+1)m}{2}}(m+1)!\sum_{i_1<\cdots<i_{m+1}} p_{i_1} \wedge \cdots  \wedge p_{i_{m+1}} \wedge q_{i_1} \wedge \cdots \wedge q_{i_{m+1}}
\end{align*}
\end{solution*}

\subsection{Вътрешно произведение}

Нека $\myxi \in V$, a $\omega^k$ e $k$-форма. Тогава $i_{\myxi}\omega^k$ е $k-1$-форма, дефинирана като $i_{\myxi}\omega^k(\myxi_1,\cdots,\myxi_{k-1})=\omega^k(\myxi,\myxi_1,\cdots,\myxi_{k-1})$. \\
Вътрешното произведение има следните свойства:
\begin{enumerate}
\item $i_{\lambda\myxi+\mu\myeta}\omega^k=\lambda i_{\myxi}\omega^k + \mu i_{\myeta}\omega^k$ \\
\begin{align*}
&i_{\lambda\myxi+\mu\myeta}\omega^k(\myxi_1,\cdots,\myxi_{k-1})=\omega^k(\lambda\myxi+\mu\myeta,\myxi_1,\cdots,\myxi_{k-1}) \\
&=\lambda\omega^k(\myxi,\myxi_1,\cdots,\myxi_{k-1})+\mu\omega^k(\myeta,\myxi_1,\cdots,\myxi_{k-1})
\end{align*}
\item $i_{\myxi}i_{\myeta}\omega^k=-i_{\myeta}i_{\myxi}\omega^k+i_{\myeta}\omega^k$
\begin{align*}
&i_{\myxi}i_{\myeta}\omega^k=i_{\myxi}\omega^k(\myeta,\myxi_1,\cdots,\myxi_{k-2}) 
=\omega^k(\myxi,\myeta,\myxi_1,\cdots,\myxi_{k-2})=-\omega^k(\myeta,\myxi,\myxi_1,\cdots,\myxi_{k-2})=\\
&-i_{\myeta}\omega^k(\myxi,\myxi_1,\cdots,\myxi_{k-2})=-i_{\myeta}i_{\myxi}\omega^k(\myxi_1,\cdots,\myxi_{k-2})
\end{align*}
Следствие: $i_{\myxi}i_{\myxi}\omega^k=0$

\item $i_{\myxi}(\omega^k \wedge \omega^l)=i_{\myxi}\omega^k \wedge \omega^l + (-1)^k \omega^k \wedge i_{\myxi}\omega^l$
\begin{align*}
&i_{\myxi}(\omega^k \wedge \omega^l)(\myxi_1,\cdots,\myxi_{k+l-1}) 
=(\omega^k \wedge \omega^l)(\myxi,\myxi_1,\cdots,\myxi_{k+l-1}) \\
&=\mathop{\sum_{i_1<\cdots<i_k}}_{j_1<\cdots<j_l} sgn(i_1,\cdots,i_k,j_1, \cdots,j_l)\omega^k(\myeta_{i_1},\cdots,\myeta_{i_k})\omega^l(\myeta_{j_1},\cdots,\myeta_{j_l}) \\
&\myeta_1=\myxi, \quad \myeta_j=\myxi_{j-1}, \enspace j=\overline{2,k+l}
\end{align*}
Ако $\myxi$ е сред аргументите на $\omega^k$ в някое събираемо, то трябва да е най-левият, тъй като $\myxi=\myeta_1, i_1<i_j, \enspace j=\overline{2,k}$. Но тогава събираемото ще е от вида
\begin{align*}
&sgn(1,\cdots,i_k,j_1, \cdots,j_l)\omega^k(\myeta_1,\cdots,\myeta_{i_k})\omega^l(\myeta_{j_1},\cdots,\myeta_{j_l}) \\
&=sgn(i_2,\cdots,i_k,j_1, \cdots,j_l) (i_{\myeta_1}\omega^k)(\myeta_{i_2},\cdots,\myeta_{i_k})\omega^l(\myeta_{j_1},\cdots,\myeta_{j_l}) \\
&=sgn(i_2,\cdots,i_k,j_1, \cdots,j_l) (i_{\myxi}\omega^k)(\myeta_{i_2},\cdots,\myeta_{i_k})\omega^l(\myeta_{j_1},\cdots,\myeta_{j_l}) 
\end{align*}
А това е точно събираемо от $i_{\myxi}\omega^k \wedge \omega^l$. От друга страна, когато $\myxi$ е аргумент на $\omega^l$, аналогично трябва да бъде първият аргумент. Това значи, че
\begin{align*}
&sgn(i_1,\cdots,i_k,1, \cdots,j_l)\omega^k(\myeta_{i_1},\cdots,\myeta_{i_k})\omega^l(\myeta_1,\cdots,\myeta_{j_l}) \\
&=(-1)^k sgn(1,i_1,\cdots,i_k, \cdots,j_l) \omega^k(\myeta_{i_1},\cdots,\myeta_{i_k})(i_{\myeta_1}\omega^l)(\myeta_{j_2},\cdots,\myeta_{j_l}) \\
&=(-1)^k sgn(1,i_1,\cdots,i_k, \cdots,j_l) \omega^k(\myxi,\cdots,\myeta_{i_k})(i_{\myeta_1}\omega^l)(\myeta_{j_2},\cdots,\myeta_{j_l}) \\
\end{align*}
Това е събираемо от $(-1)^k \omega^k \wedge i_{\myxi}\omega^l$. И тъй доказахме равенството.
\end{enumerate}

\subsection{Изображения}
Нека $\func{f}{\mathbb{R}^m}{\mathbb{R}^n}$ и $f$ е линейно изображение. Тогава
$(f^*\omega^k)(\myxi_1,\cdots,\myxi_k)=\omega^k(f(\myxi_1),\cdots,f(\myxi_k))$ е $k$-форма над $\mathbb{R}^m$, тъй като:
\begin{align*}
&(f^* \omega^k)(\myxi_1,\cdots,\lambda\myxi_i+\mu\myeta_i,\cdots,\myxi_k)=\omega^k(f(\myxi_1),\cdots,f(\lambda\myxi_i+\mu\myeta_i),\cdots, f(\myxi_k)) \\
&=\omega^k(f(\myxi_1),\cdots,\lambda f(\myxi_i)+\mu f(\myeta_i),\cdots, f(\myxi_k)) \\
&=\lambda\omega^k(f(\myxi_1),\cdots, f(\myxi_i),\cdots, f(\myxi_k))+\mu\omega^k(f(\myxi_1),\cdots, f(\myeta_i),\cdots, f(\myxi_k))\\
&(f^* \omega^k)(\myxi_{\pi_1},\cdots,\myxi_{\pi_k})=
\omega^k(f(\myxi_{\pi_1}),\cdots,f(\myxi_{\pi_k}))  =sgn(\pi)\omega^k(f(\myxi_1),\cdots,f(\myxi_k))\\
&=sgn(\pi)(f^* \omega^k)(\myxi_1,\cdots,\myxi_k) \\
\end{align*}

Може да разглеждаме $f^*$ като линеен оператор, изобразяващ векторното пространство от $k$-форми над $\mathbb{R}^m$ във векторното пространство от $k$-форми над $\mathbb{R}^n$.

\begin{align*}
&(f^*(\lambda \omega^k_1+\mu\omega^k_2))(\myxi_1,\cdots,\myxi_k)=(\lambda \omega^k_1+\mu\omega^k_2)(f(\myxi_1),\cdots, f(\myxi_k)) \\
&=(\lambda \omega^k_1)(f(\myxi_1),\cdots, f(\myxi_k)) + (\mu\omega^k_2)(f(\myxi_1),\cdots, f(\myxi_k)) \\
&=\lambda \omega^k_1(f(\myxi_1),\cdots, f(\myxi_k)) +\mu\omega^k_2(f(\myxi_1),\cdots, f(\myxi_k)) \\
&=\lambda (f^* \omega^k_1)(\myxi_1,\cdots,\myxi_k)+\mu(f^* \omega^k_2)(\myxi_1,\cdots,\myxi_k)\\
\end{align*}

Ще покажем, че $(g \circ f)^*=f^* \circ g^*$:
\begin{align*}
&((g \circ f)^*\omega^k)(\myxi_1,\cdots,\myxi_k)=\omega^k((g \circ f)(\myxi_1),\cdots, (g \circ f)(\myxi_k)) \\
&=\omega^k(g(f(\myxi_1)),\cdots, g(f(\myxi_k)))=(g^*\omega^k)(f(\myxi_1),\cdots, f(\myxi_k)) \\
&=(f^*(g^*\omega^k))(\myxi_1,\cdots, \myxi_k)= ((f^*\circ g^*)\omega^k)(\myxi_1,\cdots, \myxi_k)\\
\end{align*}

 $f*$ комутира с прилагането на външно произведение:
\begin{align*}
&(f^*(\omega^k \wedge \omega^l))(\myxi_1,\cdots,\myxi_{k+l})=(\omega^k \wedge \omega^l)(f(\myxi_1),\cdots,f(\myxi_{k+l})) \\
&=\mathop{\sum_{i_1<\cdots<i_k}}_{j_1<\cdots<j_l}sgn(i_1,\cdots,i_k,j_1, \cdots,j_l) \omega^k(f(\myxi_{i_1}),\cdots,f(\myxi_{i_k})) \omega^l(f(\myxi_{j_1}),\cdots,f(\myxi_{j_l})) \\
&=\mathop{\sum_{i_1<\cdots<i_k}}_{j_1<\cdots<j_l}sgn(i_1,\cdots,i_k,j_1, \cdots,j_l) (f^* \omega^k)(\myxi_{i_1},\cdots,\myxi_{i_k}) (f^*\omega^l)(\myxi_{j_1},\cdots,\myxi_{j_l}) \\
&=(f^* \omega^k \wedge f^*\omega^l)(\myxi_1,\cdots, \myxi_{k+l})\\
\end{align*}

\begin{comment}
&(\lambda (f^* \omega^k))(\myxi_1,\cdots,\myxi_k)=(\lambda \omega^k)(f(\myxi_1),\cdots, f(\myxi_k))=
\lambda (f^* \omega^k)(\myxi_1,\cdots,\myxi_k)
\end{comment}

Функцията $f^*$ понякога се нарича транспонирано (дуално) изображение на $f$.

\subsection{Звезда на Ходж}
Звездата на Ходж е изображение между две форми, което може да дефинираме така:
\begin{align*}
&\func{\ast}{{\bigwedge}^k}{{\bigwedge}^{n-k}} \\
&(\ast \omega) (\mathbf{e}_{j_1},\cdots, \mathbf{e}_{j_{n-k}})=sgn(i_1,\cdots,i_k,j_1,\cdots,j_{n-k})\omega(\mathbf{e}_{i_1},\cdots,\mathbf{e}_{i_k})
\end{align*}

$\ast (\omega_1+\omega_2) = \ast\omega_1+ \ast \omega_2$:
\begin{align*}
&(\ast (\omega_1+\omega_2))(\mathbf{e}_{j_1},\cdots, \mathbf{e}_{j_{n-k}})=sgn(i_1,\cdots,i_k,j_1,\cdots,j_{n-k})(\omega_1+\omega_2)(\mathbf{e}_{i_1},\cdots,\mathbf{e}_{i_k}) \\
&=sgn(i_1,\cdots,i_k,j_1,\cdots,j_{n-k})(\omega_1(\mathbf{e}_{i_1},\cdots,\mathbf{e}_{i_k})+\omega_2(\mathbf{e}_{i_1},\cdots,\mathbf{e}_{i_k})) \\
&=sgnsgn(i_1,\cdots,i_k,j_1,\cdots,j_{n-k})\omega_1(\mathbf{e}_{i_1},\cdots,\mathbf{e}_{i_k}) + sgn(i_1,\cdots,i_k,j_1,\cdots,j_{n-k})\omega_2(\mathbf{e}_{i_1},\cdots,\mathbf{e}_{i_k}) \\
&=(\ast\omega_1)(\mathbf{e}_{i_1},\cdots,\mathbf{e}_{i_k})+(\ast\omega_2)(\mathbf{e}_{i_1},\cdots,\mathbf{e}_{i_k})
\end{align*}

$\ast (\lambda\omega) = \lambda(\ast\omega)$:
\begin{align*}
&(\ast (\lambda\omega))(\mathbf{e}_{j_1},\cdots, \mathbf{e}_{j_{n-k}}) = sgn(i_1,\cdots,i_k,j_1,\cdots,j_{n-k}) (\lambda\omega)(\mathbf{e}_{i_1},\cdots,\mathbf{e}_{i_k}) \\
&=sgn(i_1,\cdots,i_k,j_1,\cdots,j_{n-k})\lambda\omega(\mathbf{e}_{i_1},\cdots,\mathbf{e}_{i_k}) \\
&=\lambda(sgn(j_1,\cdots,j_{n-k},i_1,\cdots,i_k)\omega(\mathbf{e}_{i_1},\cdots,\mathbf{e}_{i_k})) \\
&=\lambda(\ast \omega)(\mathbf{e}_{j_1},\cdots, \mathbf{e}_{j_{n-k}})
\end{align*}

$\ast \ast \omega = (-1)^{k(n-k)} \omega$:
\begin{align*}
&(\ast (\ast \omega))(e_{i_1},\cdots,e_{i_k})=sgn(j_1,\cdots,j_{n-k},i_1,\cdots,i_k)(\ast \omega)(e_{j_1},\cdots, e_{j_{n-k}}) \\
&=sgn(j_1,\cdots,j_{n-k},i_1,\cdots,i_k) sgn(i_1,\cdots,i_k,j_1,\cdots,j_{n-k})\omega(e_{i_1},\cdots,e_{i_k}) \\
&=(-1)^{k(n-k)}(sgn(i_1,\cdots,i_k,j_1,\cdots,j_{n-k}))^2\omega(e_{i_1},\cdots,e_{i_k}) \\
&=(-1)^{k(n-k)}\omega(e_{i_1},\cdots,e_{i_k})
\end{align*} 

Така звездата на Ходж в някакъв смисъл задава дуалност на $k$- форма и $n-k$-форма. Това не е толкова учудващо, тъй като пространството от $k$-форми ${\bigwedge}^k$ е с размерност $\binom{n}{k}=\binom{n}{n-k}$, размерността на  ${\bigwedge}^{n-k}$.

Има и по-обща дефиниция на $\ast$, която е свързана с изобразяване $g$ на базис на $R^n$ в положително ориентриран ортонормиран базис на $V$(разглеждат се формите над $V, dim(V)=n$), относно някакво скаларно произведение. Тогава може да се дефинира и като:
\[
\ast \omega = (g^{-1})^*(\ast g^* \omega)
\]
В този случай $\ast$  за форми над $\mathbb{R}^n$ се дефинира малко по-различно чрез билинейно изображение.


\section{Диференциални форми}
Нека $M$ е многообразие, а с $T_{\mathbf{x}}M$ бележим допирателното пространство към $M$ в точката ${\mathbf{x}}$, а $TM=\bigcup_{{\mathbf{x}} \in M}T_\mathbb{x} M$. Тъй като за произволно ${\mathbf{x}} \in \mathbb{R}^n$ е в сила $T_{\mathbf{x}} \mathbb{R}^n=\mathbb{R}^n$, то и $T\mathbb{R}^n=\mathbb{R}^n$.
Надолу под гладкост ще разбираме безкрайна гладкост.

\subsection{Диференциални 1-форми}
Диференциална $1$-форма $\omega^1$ над $M$ е гладко изображение $\func{\omega^1}{TM}{\mathbb{R}}$, линейно над всяко $T_xM$ т.е. диференциалната $1$-форма над $M$ е алгебрична $1$-форма над $T_{\mathbf{x}}M$, за всяко ${\mathbf{x}} \in M$.
Виждаме, че когато диференциалната $1$-форма $\omega$ е над $\mathbb{R}^n$, то тя е гладко линейно изображение $\stfunc{\omega}$.

Тъй като диференциалните $1$-форми са над $T_x M$, то те са в дуалното пространство, $T_x^* M$. Това пространство също се нарича котангенциално, а диференциалните $1$-формите - тангенциални (допирателни) ковектори, котангенциални вектори.

Ако $dim(T_{\mathbf{x}}M)=p$, то ако приемем, че $\dd y_i$ е проекцията на допирателен вектор в ${\mathbf{x}}$ по базисните вектори на $T_xM$, то $1$-формите се представят по следният начин:
\begin{align*}
&\omega = \sum_{i=1}^{p}g_i(\mathbf{x}) \dd y_i \\
&\omega = \sum_{i=1}^{n}g_i(\mathbf{x}) \dd y_i, TM=R^n
\end{align*}
Ако фиксираме $x$ получаваме общ запис на алгебрична $1$-форма, т.е. $\sum_{i=1}^{p}a_i \dd y_i$, т.е. самите коефициенти може да зачитаме за функция на ${\mathbf{x}}$. Тъй като в случая на форми над $R^n$ имаме, че $x \in R^n$, просто може да вземем вече наличният базис на $R^n$ и го използваме и за допирателното пространство. Така може да запишем $\omega = \sum_{i=1}^{n}g_i(\mathbf{x}) \dd x_i$.


Когато $\omega$ е над $\mathbb{R}$, то понеже е гладко изображение, то в частност е и непрекъснато. Нека $\omega=f(x)dx$. Но тогава тя е точна, понеже: 
\[
F(x)=\int_{a}^{x}f(s)ds,\quad F^\prime(x)=f(x)
\]

Това обаче не е задължително при $n>1$. За да може $\dd f=\omega$, то от $\dd f= \sum_{i=1}^{n}\pdv{f}{x_i}\dd x_i$ по теоремата на Шварц трябва да е в сила $\pdv{f}{x_i,x_j}=\pdv{f}{x_j,x_i}$. Така веднага може да видим, че $5\dd x_1 + 3x \dd x_2$ и $x_2^2 \dd x_1 + \dd x_2 - 13x_3 \dd x_3$ не са диференциал на никоя фунцкия.

\begin{problem}
Нека $\omega = - \frac{y}{x^2+y^2} \dd x + \frac{x}{x^2+y^2} \dd y$, над $\mathbb{R}^2 \setminus \{\mathbf{0}\}$, a $f(\rho,\theta)=(\rho \cos \theta,\rho \sin \theta)^T$. Да се намери $f^* \omega$.
\end{problem}

\begin{solution*}
\begin{align*}
&(f^* \omega)(\rho, \theta) = \omega(f_1(\rho,\theta),f_2(\rho,\theta)) \\
&= - \frac{\rho \sin \theta}{(\rho \cos \theta)^2+(\rho \sin \theta)^2} \dd (\rho \cos \theta) + \frac{\rho \cos \theta}{(\rho \cos \theta)^2+(\rho \sin \theta)^2} \dd (\rho \sin \theta) \\
&=- \frac{\rho \sin \theta}{\rho^2} (\cos \theta \dd \rho - \rho \sin \theta \dd \theta)+\frac{\rho \cos \theta}{\rho^2}(\sin \theta \dd \rho + \rho \cos \theta \dd \theta) \\
&=-\frac{\sin \theta \cos \theta}{\rho} \dd \rho + \sin^2 \theta \dd \theta + \frac{\cos \theta \sin \theta}{\rho} \dd \rho + \cos^2 \theta \dd \theta  = (\sin^2 \theta + \cos^2 \theta) \dd \theta = \dd \theta
\end{align*}
\end{solution*}

\subsection{Диференциални k-форми}

Уточнението, което направихме по-рано сега влиза в общата дефиниция.
Диференциална $k$-форма $\omega \vert_x$ в точка $x \in M$ е (външна) $k$-форма над $T_x M$. Ако тя съществува за кое да е $x \in M$ и е диференцируема в него, то казваме, че формата е над $M$ и бележим само $\omega$.

Тъй като дефинираме диференциалните $k$-форми като вид $k$-форми, то и за тях са дефинирани операциите външно и вътрешно произведение.



\begin{problem}
Да се пресметне $\omega \vert_x= \dd x_1 \wedge \dd (x_1^2+x_2^2+x_3^2)$ в точката $x=(1,2,3)^T$ с аргументи допирателните вектори $\myxi_1=(0,5,2)^T,\myxi_2=(3,2,1)^T$
\end{problem}

\begin{solution*}
\begin{align*}
&\dd (x_1^2+x_2^2+x_3^2) = 2x_1 \dd x_1 + 2x_2 \dd x_2 + 2x_3 \dd x_3 \\
&\dd x_1 \wedge \dd (x_1^2+x_2^2+x_3^2) = 2x_2 \dd x_1 \wedge \dd x_2 + 2x_3 \dd x_1 \wedge \dd x_3 \\
&\omega \vert_x = 4\dd x_1 \wedge \dd x_2 + 6\dd x_1 \wedge \dd x_3 \\
&\omega(\myxi_1,\myxi_2) = 4\dd x_1 \wedge \dd x_2(\myxi_1,\myxi_2) + 6\dd x_1 \wedge \dd x_3(\myxi_1,\myxi_2) \\
&=4 \begin{vmatrix}
0 &3 \\
5 &2
\end{vmatrix}  + 6 \begin{vmatrix}
0 &3 \\
2 &1
\end{vmatrix} = 4(-15)+6(-6)=-60-36=-96 \\
\end{align*}
\end{solution*}



\subsection{Външно диференциране}
Външното диференциране е допълнителна операция над диференциалните форми, даваща ни диференциална форма от степен с едно повече.

В зависимост от дефинцията следните свойства или са аксиоматични или се извеждат:
\begin{enumerate}
\item $\dd$ е линейно изображение.
\item $\dd f$ е диференциал на $f$, ако $f$ е гладка фунцкия($0$-форма)
\item $\dd (\dd f)=0$, ако $f$ е $0$-форма
\item $\dd (\omega^k \wedge \omega^l)=\dd \omega^k \wedge \omega^l + (-1)^k \omega^k \wedge \dd \omega^l$
\end{enumerate}

Няколко основни свойства:
\begin{enumerate}
\begin{comment}
\item $\dd (\omega_1+\omega_2)=\dd \omega_1 + \dd \omega_2$
По индукция:
\begin{align*}
&\omega_1,\omega_2 \in {\bigwedge}^0: \quad  \omega_1=f, \enspace \omega_2=g,  
\dd (\omega_1+\omega_2)=\dd (f+g)=\dd f + \dd g = \dd \omega_1 + \dd \omega_2 \\
&\omega_1, \omega_2 \in {\bigwedge}^k: \quad d(\omega_1+\omega_2)=\dd(\sum_{i_1<\cdots<i_k}a_{i_1 \cdots i_k}(\mathbf{x}) \dd x_{i_1} \wedge \cdots \dd x_{i_k}+\sum_{i_1<\cdots<i_k}b_{i_1 \cdots i_k}(\mathbf{x}) \dd x_{i_1} \wedge \cdots \dd x_{i_k}) \\
&=\dd (\sum_{i_1<\cdots<i_k}(a_{i_1 \cdots i_k}(\mathbf{x})+b_{i_1 \cdots i_k}(\mathbf{x})) \dd x_{i_1} \wedge \cdots \dd x_{i_k}) \\
&=
\end{align*}
\end{comment}
\item $\dd \dd = 0$
\begin{align*}
&\omega \in {\bigwedge}^k \implies \omega = \sum_{i_1<\cdots<i_k}a_{i_1 \cdots i_k}(\mathbf{x}) \dd x_{i_1} \wedge \cdots \dd x_{i_k} \\
&\dd (\dd \omega) = \dd(\sum_{i_1<\cdots<i_k}\dd(a_{i_1 \cdots i_k}(\mathbf{x}) \dd x_{i_1} \wedge \cdots \dd x_{i_k})) \\
&= \dd (\sum_{i_1<\cdots<i_k} \dd a_{i_1 \cdots i_k}(\mathbf{x}) \wedge \dd x_{i_1} \wedge \cdots \dd x_{i_k} + \sum_{i_1<\cdots<i_k} a_{i_1 \cdots i_k}(\mathbf{x}) \dd(\dd x_{i_1} \wedge \cdots \dd x_{i_k})) \\
&=\sum_{i_1<\cdots<i_k}\dd(\dd a_{i_1 \cdots i_k}(\mathbf{x}) \wedge \dd x_{i_1} \wedge \cdots \dd x_{i_k}) \\ 
&+\dd (\sum_{i_1<\cdots<i_k} a_{i_1 \cdots i_k}(\mathbf{x}) \dd \dd x_{i_1} \wedge \dd x_{i_2} \cdots \dd x_{i_k}-\dd x_{i_1} \wedge \dd( \dd x_{i_2} \wedge \cdots \wedge \dd x_{i_k})) \quad ddx_j=0 \implies \\
&=\sum_{i_1<\cdots<i_k} \dd \dd a_{i_1 \cdots i_k}(\mathbf{x}) \wedge \dd x_{i_1} \wedge \cdots \dd x_{i_k} - \dd a_{i_1 \cdots i_k}(\mathbf{x}) \wedge \dd(\dd x_{i_1} \wedge \cdots \dd x_{i_k}) + \dd(0) \\
&=0 + 0 = 0 \quad (\dd \dd a_{i_1 \cdots i_k}(\mathbf{x})=0)
\end{align*}
Така видяхме и че:
\begin{align*}
\dd (a_{i_1 \cdots i_k}(\mathbf{x}) \dd x_{i_1} \wedge \cdots \dd x_{i_k})=\dd a_{i_1 \cdots i_k}(\mathbf{x}) \wedge \dd x_{i_1} \wedge \cdots \dd x_{i_k}=\sum_{j=1}^{n} \pdv{a_{i_1 \cdots i_k}}{x_j}(\mathbf{x}) \dd x_j \wedge \dd x_{i_1} \wedge \cdots \dd x_{i_k}
\end{align*} 
\item $f^*(\dd \omega)=\dd(f^* \omega)$
\begin{align*}
%&f^*(\dd \omega)(\myxi_1,\cdots,\myxi_{k+1})= (\dd \omega)(f(\myxi_1),\cdots,f(\myxi_{k+1}))= \sum_{i_1<\cdots<i_k} \dd a_{i_1 \cdots i_k}() \\
%&\dd(f^* \omega) = \dd ( \sum_{i_1<\cdots<i_k} (a_{i_1 \cdots i_k} \circ f) \dd f_{i_1}  \wedge \dd f_{i_1} ) \\
%&=\dd (\sum_{i_1<\cdots<i_k} \sum_{j=1}^{n} \pdv{(a_{i_1 \cdots i_k} \circ f)}{x_j} \dd x_j \wedge \dd f_{i_1}  \wedge \dd f_{i_1} ) \\
%&=\dd (\sum_{i_1<\cdots<i_k} \sum_{j=1}^{n} \sum_{s=1}^{n} (\pdv{a_{i_1 \cdots i_k}}{x_s} \circ f) \pdv{f_s}{x_j} \dd x_j \wedge \dd f_{i_1}  \wedge \dd f_{i_1} ) \\
&f^*(\dd \omega) = \sum_{i_1<\cdots<i_k} ((\dd a_{i_1 \cdots i_k}) \circ f) \wedge \dd f_{i_1} \wedge \cdots  \wedge \dd f_{i_k} = \sum_{i_1<\cdots<i_k} (\dd( a_{i_1 \cdots i_k} \circ f)) \wedge \dd f_{i_1} \wedge \cdots  \wedge \dd f_{i_k} \\
&= \dd (\sum_{i_1<\cdots<i_k} ( a_{i_1 \cdots i_k} \circ f) \dd f_{i_1} \wedge \cdots  \wedge \dd f_{i_k}) = d(f^* \omega)
\end{align*}
\end{enumerate}

\begin{problem}
Да се намерят $\dd \omega, \dd^2 \omega$, ако $\omega = z^3 \dd x \wedge \dd y + (z^2+2y) \dd x \wedge \dd z$
\end{problem}

\begin{solution*}
\begin{align*}
&\dd \omega = \dd (z^3 \dd x \wedge \dd y + (z^2+2y) \dd x \wedge \dd z) = \dd z^3 \wedge \dd x \wedge \dd y + \dd(z^2+2y) \wedge \dd x \wedge \dd z \\
&=3z^2 \dd z \wedge \dd x \wedge \dd y + 2z \dd z \wedge \dd x \wedge \dd z + 2 \dd y \wedge \dd x \wedge \dd z =3z^2 \dd x \wedge \dd y \wedge \dd z - 2 \dd x \wedge \dd y \wedge \dd z \\
&=(3z^2-2)\dd x \wedge \dd y \wedge \dd z \\
&\dd \dd \omega = 6 z \dd z \wedge \dd x \wedge \dd y \wedge \dd z = 0
\end{align*}
\end{solution*}

\begin{comment}
\begin{problem}
Докажете, че $\dd \dd f = 0$
\end{problem}

\begin{solution*}
\begin{align*}
&\dd (\dd f) = \dd (\sum_{i=1}^{n} \pdv{f}{x_i} \dd x_i ) = \sum_{i=1}^{n} (\sum_{j=1}^{n} \pdv{f}{x_j, x_i} \dd x_j) \wedge \dd x_i = \sum_{i=1}^{n} \sum_{j\neq i} \pdv{f}{x_j, x_i} \dd x_j \wedge \dd x_i \\
&= \sum_{i=1}^{n} \sum_{j<i} \pdv{f}{x_j, x_i} \dd x_j \wedge \dd x_i + \sum_{i=1}^{n} \sum_{j>i} \pdv{f}{x_j, x_i} \dd x_j \wedge \dd x_i  \\
&= \sum_{i=1}^{n} \sum_{j<i} \pdv{f}{x_j, x_i} \dd x_j \wedge \dd x_i + \sum_{j=1}^{n} \sum_{i<j} \pdv{f}{x_j, x_i} \dd x_j \wedge \dd x_i \\
&= \sum_{i=1}^{n} \sum_{j<i} -\pdv{f}{x_j, x_i} \dd x_i \wedge \dd x_j + \sum_{i=1}^{n} \sum_{j<i} \pdv{f}{x_i, x_j} \dd x_i \wedge \dd x_j \\
&= \sum_{i=1}^{n} \sum_{j<i}(\pdv{f}{x_i, x_j}-\pdv{f}{x_j, x_i}) \dd x_i \wedge \dd x_j = 0 \text{(от теоремата на Шварц)}
\end{align*}
\end{solution*}
\end{comment}


Казваме, че диференциална форма $\alpha$ е точна, когато $\alpha=\dd \beta$.
Казваме, че диференциална форма $\omega$ е затворена, ако $\dd \omega=0$.

Точните форми са затворени:
\begin{align*}
&\omega = \dd \beta \implies \dd \omega = \dd \dd \beta = 0 
\end{align*}

\begin{problem}
Разгледайте $\omega = \sum_{i=1}^n \dd p_i \wedge \dd q_i$, над $\mathbb{R}^{2n}=(\mathbf{q},\mathbf(p))$ и докажете, че е точна и затворена.
\end{problem}

\begin{solution*}
\begin{align*}
&\alpha = \sum_{i=1}^n p_i \dd q_i , \quad \dd \alpha =\dd(\sum_{i=1}^n p_i \dd q_i) = \sum_{i=1}^n \dd(p_i \dd q_i)
= \sum_{i=1}^n \dd p_i \wedge \dd q_i = \omega \\
&\dd \omega = \dd(\sum_{i=1}^n \dd p_i \wedge \dd q_i) =\sum_{i=1}^n \dd(\dd p_i \wedge \dd q_i) = 
\sum_{i=1}^n  \dd 1 \wedge \dd p_i \wedge \dd q_i = \sum_{i=1}^n 0 = 0
\end{align*}
\end{solution*}

\begin{lemma}[Лема на Поанкаре]
Всяка затворена диференциална форма над $R^n$ е точна.
\end{lemma}

\subsection{Кодиференциал}
Кодиференциалът е изображение от диференциални форми в такива от един ред по-нисък.
\[
\delta \omega=(-1)^{n(k+1)+1}\ast \dd \ast \omega, \quad \omega \in {\bigwedge}^k
\]

Виждаме, че $\delta^2=0$, тъй като:
\begin{align*}
&\delta \delta \omega^k = \delta (-1)^{n(k+1)+1}\ast \dd \ast \omega^k = (-1)^{n(k+1)+1} (-1)^{nk+1} \ast \dd \ast \ast \dd \ast \omega^k \\
&= (-1)^{n(2k+1)} (-1)^{n(n-k+1)} \ast \dd \dd \ast \omega^k = (-1)^{n(n+k+2)} \ast \dd \dd \alpha^{n-k} = (-1)^{n(n+k+2)} \ast 0 = 0\\
\end{align*}
Ако $f$ е гладка фунцкия, то: 
\begin{align*}
&\delta f = (-1)^{n+1} \ast \dd \ast f = (-1)^{n+1} \ast \dd(f \dd x_1 \wedge \cdots \wedge \dd x_n) = (-1)^{n+1} \ast (\dd f \wedge \dd x_1 \wedge \cdots \wedge \dd x_n) \\
&=(-1)^{n+1} \ast (\sum_{i=1}^{n} \pdv{f}{x_i} \dd x_i \wedge \dd x_1 \wedge \cdots \wedge \dd x_n) = (-1)^{n+1} \ast 0 = 0
\end{align*}

\subsection{Диференциал на Ли}
Нека $\mathbf{V}$ е векторно поле, а $f$-гладка фунцкия. Тогава под диференциал на Ли разбираме:
\begin{align*}
&L_\mathbf{V} = \sum_{i=1}^{n}\mathbf{V}_i(\mathbf{x})\pdv{}{x_i} \\
&L_\mathbf{V}(f)=\sum_{i=1}^{n}\mathbf{V}_i(\mathbf{x})\pdv{f}{x_i}
\end{align*}

Лесно се вижда, че за $0$-форма, т.е. гладка фунцкия $f$ е в сила:
\begin{align*}
&L_\mathbf{V}(f)=\sum_{i=1}^{n}\mathbf{V}_i(\mathbf{x})\pdv{f}{x_i}= \dotpr{\mathbf{V}}{\mathbf{grad} f}
\end{align*}

Веднага се вижда, че:
\begin{align*}
&L_{\mathbf{V}+\mathbf{W}}= \sum_{i=1}^{n}(\mathbf{V}+\mathbf{W})_i(\mathbf{x})\pdv{}{x_i} = \sum_{i=1}^{n}(\mathbf{V}_i(\mathbf{x})+\mathbf{W}_i(\mathbf{x}))\pdv{}{x_i} \\
&=\sum_{i=1}^{n}\mathbf{V}_i(\mathbf{x})\pdv{}{x_i}+\sum_{i=1}^{n}\mathbf{W}_i(\mathbf{x})\pdv{}{x_i} = L_{\mathbf{V}} + L_{\mathbf{W}} \\
&L_{g\mathbf{V}}= \sum_{i=1}^{n}(g\mathbf{V})_i(\mathbf{x})\pdv{}{x_i} = \sum_{i=1}^{n}g(\mathbf{x})\mathbf{V}_i(\mathbf{x})\pdv{}{x_i} = g(\mathbf{x})\sum_{i=1}^{n}\mathbf{V}_i(\mathbf{x})\pdv{}{x_i} = g L_{\mathbf{V}} \\
\end{align*}

В зависимост от дефиницията на $L_\mathbf{V}$ следната формула на Картан(известна още като формула на хомотопията) или се извежда, или може да се приеме за дефиниция на действието на диференциала на Ли над форми:
\begin{align*}
&L_\mathbf{V}\omega=i_V \dd \omega + \dd i_V \omega
\end{align*}
Тоест с диференциала на Ли получаваме диференциална форма от същия ред.


\begin{problem}
Нека $\omega = \sum_{i=1}^n \dd p_i \wedge \dd q_i$, над $\mathbb{R}^{2n}=(\mathbf{q},\mathbf{p})$ и дефинираме следното векторно поле: $\mathbf{X}_H=(\pdv{H}{p_1},\cdots,\pdv{H}{p_i},\cdots,\pdv{H}{p_n},-\pdv{H}{q_1},\cdots,-\pdv{H}{q_i},\cdots,-\pdv{H}{q_n})^T$. Да се намерят $L_{\mathbf{X}_H}H$ и $L_{\mathbf{X}_H}\omega$.
\end{problem}

\begin{solution*}
\begin{align*}
&L_{\mathbf{X}_H}=\sum_{i=1}^n \pdv{H}{p_i}\pdv{}{q_i}-\pdv{H}{q_i}\pdv{}{p_i} \\
&L_{\mathbf{X}_H} H = i_{\mathbf{X}_H} \dd H + \dd i_{\mathbf{X}_H} H = i_{\mathbf{X}_H} \dd H = \dd H(\mathbf{X}_H) = \sum_{i=1}^n \pdv{H}{q_i} \dd q_i(\mathbf{X}_H) + \pdv{H}{p_i} \dd p_i(\mathbf{X}_H) \\
&=\sum_{i=1}^n \pdv{H}{q_i} (\mathbf{X}_H)_{q_i} + \pdv{H}{p_i}  (\mathbf{X}_H)_{p_i} = \sum_{i=1}^n \pdv{H}{q_i} \pdv{H}{p_i} + \pdv{H}{p_i} (-\pdv{H}{q_i}) = 0 \\
&L_{\mathbf{X}_H} \omega = i_{\mathbf{X}_H} \dd \omega + \dd i_{\mathbf{X}_H} \omega = \dd i_{\mathbf{X}_H} \omega = \dd (\sum_{i=1}^n \dd p_i \wedge \dd q_i (\mathbf{X}_H,\myxi)) \\
&= \dd (\sum_{i=1}^n (\mathbf{X}_H)_{p_i} \dd q_i(\myxi) - \dd p_i(\myxi) (\mathbf{X}_H)_{q_i} )  \\
&=\dd(\sum_{i=1}^n -\pdv{H}{q_i} \dd q_i(\myxi) - \pdv{H}{p_i} \dd p_i(\myxi)) = - \dd(\sum_{i=1}^n \pdv{H}{q_i} \dd q_i + \pdv{H}{p_i} \dd p_i)(\myxi) = - \dd \dd H (\myxi) = 0
\end{align*}
\end{solution*}
$L_{\mathbf{X}_H} H = 0$ може да се покаже и направо със заместване.

\subsection{Аналози на векторни операции}
Външната производна и оператора на Ходж могат да се използват в съвкупност и така да се получат повечето известни от математическия анализ "важни" функции на векторните полета.
За следващите определения предполагаме $V$ поле в $\mathbb{R}^n$, а $f$ гладка фунцкия.

\subsubsection*{Градиент}
Градиентът на $f$ ни дава векторното поле $\mathbf{grad} f=(\pdv{f}{x_1},\cdots,\pdv{f}{x_n})^T$. Той е обвързан с диференцирането по посока (направление) $\mathbf{w} \in \mathbb{R}^n$.
\[
\dd f(n)=\sum_{i=1}^{n} \pdv{f}{x_i}dx_i(w)=\sum_{i=1}^{n} \pdv{f}{x_i}w_i=\dotpr{\mathbf{grad}f}{\mathbf{w}}
\] 
Максималната стойност на горното скаларно произведение се достига при $\mathbf{w}=\mathbf{grad}f$, откъдето и $\mathbf{grad}f$ е посоката на най-голямо нарастване на фунцкията. И обратно, най-малката стойност се достига при $\mathbf{w}=-\mathbf{grad}f$ - посоката на най-голямо намаляване. Тези свойства се използват при някои оптимизационни алгоритми, базиране на предвижване по тези посоки, за да се достигне съответно локален максимум или минимум.
Градиентът също се бележи и с $\grad f$, а понякога и директно с $f^\prime$ за показване на аналог на формули при по-писока размерност.

\subsubsection*{Дивергенция}
Дивергенцията на векторното поле $\mathbf{V}$ му съпоставя функция :
\[
div \mathbf{V} = \sum_{i=1}^{n} \pdv{\mathbf{V_i}}{x_i}
\] 
Интуитивно, дивергенцията ни дава идея дали дали векторното поле "сочи към" или "сочи далеч" от дадена точка. Тя е свързана с $n$-формата $\dd (\ast \omega)$, където $\omega$ е съпоставената по полето $1$-форма:
\begin{align*}
&\dd (\ast \omega) = \dd ( \ast \sum_{i=1}^{n} \mathbf{V_i} \dd x_i ) = \dd( \sum_{i=1}^{n} \mathbf{V_i} \ast \dd x_i ) \\
&= \dd(\sum_{i=1}^{n} \mathbf{V_i} sgn(i,1,\cdots,i-1,i+1,\cdots,n) \dd x_1 \wedge \cdots \dd x_{i-1} \wedge \dd x_{i+1} \wedge \cdots \wedge x_n) \\
&=\sum_{i=1}^{n} \dd(\mathbf{V_i}) (-1)^{i-1} \wedge \dd x_1 \wedge \cdots \dd x_{i-1} \wedge \dd x_{i+1} \wedge \cdots \wedge x_n \\
&=\sum_{i=1}^{n} (-1)^{i-1}  (\sum_{j=1}^{n} \pdv{\mathbf{V_j}}{x_j} \dd x_j ) \wedge \dd x_1 \wedge \cdots \dd x_{i-1} \wedge \dd x_{i+1} \wedge \cdots \wedge x_n \\
&=\sum_{i=1}^{n} (-1)^{i-1} \pdv{\mathbf{V_i}}{x_i} \dd x_i  \wedge \dd x_1 \wedge \cdots \dd x_{i-1} \wedge \dd x_{i+1} \wedge \cdots \wedge x_n) \\
&=\sum_{i=1}^{n} (-1)^{i-1}(-1)^{i-1} \pdv{\mathbf{V_i}}{x_i}  \dd x_1 \wedge \cdots \wedge x_n \\
&=(\sum_{i=1}^{n} \pdv{\mathbf{V_j}}{x_i})  \dd x_1 \wedge \cdots \wedge x_n = (div \mathbf{V}) \dd \mu_n \\
&\ast \dd (\ast \omega) = \ast (div \mathbf{V}) \dd \mu_n = div \mathbf{V} sgn(1,\cdots,n) = div \mathbf{V}
\end{align*}


Дивергенцията също се бележи и с $\div \mathbf{V}$(чисто мнемоническо означение, тъй като $\nabla=(\pdv{}{x_1},\cdots,\pdv{}{x_n})$ не е истински вектор).

\subsubsection*{Ротор}
Роторът на векторното поле $\mathbf{V}$ му съпоставя $n-2$-форма:
\[
\mathbf{rot} \mathbf{V} = \ast \dd \omega, \quad 
\omega=\sum_{i=1}^n \mathbf{V}_i \dd x_i
\] 

Роторът също се бележи и с $\mathbf{curl} \mathbf{V}$ или $\curl \mathbf{V}$ (отново за лесно запомняне, но това важи само за $\mathbb{R}^3$). Роторът може да се срещне и като "завихряне".

Използвайки, че $d^2=0$, получаваме формулатите $\mathbf{rot} \, \mathbf{grad}f=\ast \dd \dd f = \ast 0 = 0$ и $div \, \mathbf{rot} \mathbf{V}=\ast \dd \ast \ast \dd \omega = (-1)^{0n}\ast \dd \dd \omega = 0$

Нека фиксираме векторни полета $\mathbf{A}$ и $\mathbf{B}$ над $\mathbb{R}^3$ и бележим $\omega_{\mathbf{A}}^1=A_1 \dd x_1 + A_2 \dd x_2 + A_3 \dd x_3, \omega_{\mathbf{A}}^2=A_1 \dd x_2 \wedge \dd x_3 + A_2 \dd x_3 \wedge \dd x_1 + A_3 \dd x_1 \wedge \dd x_2$(аналогично дефинираме и за $\mathbf{B}$). Тогава:

\begin{align*}
&\omega_{\mathbf{A}}^1 \wedge \omega_{\mathbf{B}}^1 = (A_1 \dd x_1 + A_2 \dd x_2 + A_3 \dd x_3) \wedge (B_1 \dd x_1 + B_2 \dd x_2 + B_3 \dd x_3) \\
&= A_1 B_2 \dd x_1 \wedge \dd x_2 + A_1 B_3 \dd x_1 \wedge \dd x_3 + A_2 B_1 \dd x_2 \wedge \dd x_1 + A_2 B_3 \dd x_2 \wedge \dd x_3 \\
&+ A_3 B_1 \dd x_3 \wedge \dd x_1 + A_3 B_2 \dd x_3 \wedge \dd x_2 \\
&=(A_2 B_3 - A_3 B_2)\dd x_2 \wedge \dd x_3 + (A_3 B_1 - A_1 B_3)\dd x_3 \wedge \dd x_1 + (A_1 B_2 - A_2 B_1)\dd x_3 \wedge \dd x_2 \\
&= \omega_{\mathbf{A} \times \mathbf{B}}^2 \\
&\omega_{\mathbf{A}}^1 \wedge \omega_{\mathbf{B}}^2 = (A_1 \dd x_1 + A_2 \dd x_2 + A_3 \dd x_3) \wedge (B_1 \dd x_2 \wedge \dd x_3 + B_2 \dd x_3 \wedge \dd x_1 + B_3 \dd x_1 \wedge \dd x_2) \\
&= A_1 B_1 \dd x_1 \wedge \dd x_2 \wedge \dd x_3  +  A_2 B_2 \dd x_2 \wedge \dd x_3 \wedge \dd x_1  + A_3 B_3 \dd x_3 \wedge \dd x_1 \wedge \dd x_2 \\
&=A_1 B_1 \dd x_1 \wedge \dd x_2 \wedge \dd x_3 + A_2 B_3 \dd x_1 \wedge \dd x_2 \wedge \dd x_3 + A_3 B_3 \dd x_1 \wedge \dd x_2 \wedge \dd x_3 \\
&=\dotpr{\mathbf{A}}{\mathbf{B}} \dd x_1 \wedge \dd x_2 \wedge \dd x_3 = \dotpr{\mathbf{A}}{\mathbf{B}} \dd \mu_3
\end{align*}

\begin{problem}
Да се намерят векторни операции, съответствия на $\dd f$, $\dd \omega_{\mathbf{A}}^1$ и $\dd \omega_{\mathbf{A}}^2$ 
\end{problem}

\begin{solution*}
\begin{align*}
&\dd f = \pdv{f}{x_1} \dd x_1 + \pdv{f}{x_2} \dd x_2 + \pdv{f}{x_3} \dd x_3 = \omega_{\mathbf{grad}f}^1 \\
&\dd \omega_{\mathbf{A}}^1 = \dd (A_1 \dd x_1 + A_2 \dd x_2 + A_3 \dd x_3) = \dd A_1 \dd x_1 + \dd A_2 \dd x_2 + \dd A_3 \dd x_3 \\
&=\pdv{A_1}{x_2} \dd x_2 \wedge \dd x_1 + \pdv{A_1}{x_3} \dd x_3 \wedge \dd x_1 + \pdv{A_2}{x_1} \dd x_1 \wedge \dd x_2 \\ 
&+\pdv{A_2}{x_3} \dd x_3 \wedge \dd x_1 + \pdv{A_3}{x_1} \dd x_1 \wedge \dd x_3 + \pdv{A_3}{x_2} \dd x_2 \wedge \dd x_3 \\
&= \left(\pdv{A_3}{x_2} - \pdv{A_2}{x_3}\right) \dd x_2 \wedge \dd x_3 + \left(\pdv{A_1}{x_3} - \pdv{A_3}{x_1}\right) \dd x_3 \wedge \dd x_1 + \left(\pdv{A_2}{x_1} - \pdv{A_1}{x_2}\right) \dd x_1 \wedge \dd x_2 \\
&=\omega_{\mathbf{rot}\mathbf{A}}^2 \\
&\dd \omega_{\mathbf{A}}^2 = \dd (A_1 \dd x_2 \wedge \dd x_3 + A_2 \dd x_3 \wedge \dd x_1 + A_3 \dd x_1 \wedge \dd x_2) \\
&= \dd A_1 \dd x_2 \wedge \dd x_3 + \dd A_2 \dd x_3 \wedge \dd x_1 + \dd A_3 \dd x_1 \wedge \dd x_2 \\
&=\pdv{A_1}{x_1} \dd x_1 \wedge \dd x_2 \wedge \dd x_3 + \pdv{A_2}{x_2} \dd x_2 \wedge \dd x_3 \wedge \dd x_1 + \pdv{A_3}{x_3} \dd x_3 \wedge \dd x_1 \wedge \dd x_2 \\
&=\left(\pdv{A_1}{x_1}+\pdv{A_2}{x_2}+\pdv{A_3}{x_3}\right)\dd x_1 \wedge \dd x_2 \wedge \dd x_3 = (div \mathbf{A})d\mu_3 
\end{align*}
\end{solution*}

\subsection{Оператор на Лаплас-Белтрами-де Рам}
Този операторът на Лаплас-де Рам се дефинира така:
\[
\Delta = (\dd + \delta)^2 = \dd \delta + \delta \dd
\]
До формулата отдясно стигаме след използване на $d^2=\delta^2=0$. Формите $\omega$, за които $\Delta \omega = 0$ се наричат хармонични. 
Известният ни Лапласиан е операторът на Лаплас-Белтрами:
\[
\Delta f = div \, \mathbf{grad} f
\]
Обобщението на Лаплас-Белтрами е Лаплас-де Рам. Ако то е приложено на фунцкия, дава $\Delta f = \dd \delta + \delta \dd f = \delta \dd f$. Може да се покаже, че $\delta \dd f=-\Delta f$(в смисълът на Лаплас-Белтрами). Това не е проблем, тъй като хармонични $0$-форми, т.е. фунцкии са едни и същи в двата случая (от $-\Delta f = 0$ $\Longleftrightarrow$ $\Delta f = 0$).

\begin{problem}
Покажете,че наистина $\delta \dd f = - div \, \mathbf{grad} f$
\end{problem}

\begin{proof}
Тъй като $f$ е гладка фунцкия, то може да намерим неиният градиент, $\mathbf{grad} f$. Тъй като той е векторно поле, то $\ast \dd \ast \omega = div \, \mathbf{grad} f$(тук $\omega$ е съответстващата на градиента $1$-форма). Но $\dd f = \omega$. Тогава $\delta \dd f = (-1)^{n(1+1)+1} \ast \dd \ast \dd f = (-1) \ast \dd \ast \omega = - div \, \mathbf{grad} f$ 
\end{proof}

\subsection{Интегриране на диференциални форми}

\begin{theorem*}[Формула на Стокс]
 Нека $M$ е компактно гладко ориентируемо многообразие с непразна граница $\partial M \subset \mathbb{R}^n$. Нека $\omega$ е диференциална $n-1$-форма. Тогава $\int_{\partial M} \omega = \int_M \dd \omega$.
\end{theorem*}

Теоремата на Стокс е обобщение на теоремите на Нютон-Лайбниц, Грийн и Гаус-Остроградски.

\end{large}
\end{document}
